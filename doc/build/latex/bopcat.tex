% Generated by Sphinx.
\def\sphinxdocclass{report}
\documentclass[letterpaper,10pt,english]{sphinxmanual}

\usepackage[utf8]{inputenc}
\ifdefined\DeclareUnicodeCharacter
  \DeclareUnicodeCharacter{00A0}{\nobreakspace}
\else\fi
\usepackage{cmap}
\usepackage[T1]{fontenc}
\usepackage{amsmath,amssymb}
\usepackage{babel}
\usepackage{times}
\usepackage[Bjarne]{fncychap}
\usepackage{longtable}
\usepackage{sphinx}
\usepackage{multirow}
\usepackage{eqparbox}


\addto\captionsenglish{\renewcommand{\figurename}{Fig. }}
\addto\captionsenglish{\renewcommand{\tablename}{Table }}
\SetupFloatingEnvironment{literal-block}{name=Listing }

\addto\extrasenglish{\def\pageautorefname{page}}


\usepackage{enumitem}
\setlistdepth{99}

\title{BOPcat Documentation}
\date{Sep 26, 2018}
\release{1.0}
\author{Alvin Noe Ladines}
\newcommand{\sphinxlogo}{}
\renewcommand{\releasename}{Release}
\makeindex

\makeatletter
\def\PYG@reset{\let\PYG@it=\relax \let\PYG@bf=\relax%
    \let\PYG@ul=\relax \let\PYG@tc=\relax%
    \let\PYG@bc=\relax \let\PYG@ff=\relax}
\def\PYG@tok#1{\csname PYG@tok@#1\endcsname}
\def\PYG@toks#1+{\ifx\relax#1\empty\else%
    \PYG@tok{#1}\expandafter\PYG@toks\fi}
\def\PYG@do#1{\PYG@bc{\PYG@tc{\PYG@ul{%
    \PYG@it{\PYG@bf{\PYG@ff{#1}}}}}}}
\def\PYG#1#2{\PYG@reset\PYG@toks#1+\relax+\PYG@do{#2}}

\expandafter\def\csname PYG@tok@gd\endcsname{\def\PYG@tc##1{\textcolor[rgb]{0.63,0.00,0.00}{##1}}}
\expandafter\def\csname PYG@tok@gu\endcsname{\let\PYG@bf=\textbf\def\PYG@tc##1{\textcolor[rgb]{0.50,0.00,0.50}{##1}}}
\expandafter\def\csname PYG@tok@gt\endcsname{\def\PYG@tc##1{\textcolor[rgb]{0.00,0.27,0.87}{##1}}}
\expandafter\def\csname PYG@tok@gs\endcsname{\let\PYG@bf=\textbf}
\expandafter\def\csname PYG@tok@gr\endcsname{\def\PYG@tc##1{\textcolor[rgb]{1.00,0.00,0.00}{##1}}}
\expandafter\def\csname PYG@tok@cm\endcsname{\let\PYG@it=\textit\def\PYG@tc##1{\textcolor[rgb]{0.25,0.50,0.56}{##1}}}
\expandafter\def\csname PYG@tok@vg\endcsname{\def\PYG@tc##1{\textcolor[rgb]{0.73,0.38,0.84}{##1}}}
\expandafter\def\csname PYG@tok@vi\endcsname{\def\PYG@tc##1{\textcolor[rgb]{0.73,0.38,0.84}{##1}}}
\expandafter\def\csname PYG@tok@mh\endcsname{\def\PYG@tc##1{\textcolor[rgb]{0.13,0.50,0.31}{##1}}}
\expandafter\def\csname PYG@tok@cs\endcsname{\def\PYG@tc##1{\textcolor[rgb]{0.25,0.50,0.56}{##1}}\def\PYG@bc##1{\setlength{\fboxsep}{0pt}\colorbox[rgb]{1.00,0.94,0.94}{\strut ##1}}}
\expandafter\def\csname PYG@tok@ge\endcsname{\let\PYG@it=\textit}
\expandafter\def\csname PYG@tok@vc\endcsname{\def\PYG@tc##1{\textcolor[rgb]{0.73,0.38,0.84}{##1}}}
\expandafter\def\csname PYG@tok@il\endcsname{\def\PYG@tc##1{\textcolor[rgb]{0.13,0.50,0.31}{##1}}}
\expandafter\def\csname PYG@tok@go\endcsname{\def\PYG@tc##1{\textcolor[rgb]{0.20,0.20,0.20}{##1}}}
\expandafter\def\csname PYG@tok@cp\endcsname{\def\PYG@tc##1{\textcolor[rgb]{0.00,0.44,0.13}{##1}}}
\expandafter\def\csname PYG@tok@gi\endcsname{\def\PYG@tc##1{\textcolor[rgb]{0.00,0.63,0.00}{##1}}}
\expandafter\def\csname PYG@tok@gh\endcsname{\let\PYG@bf=\textbf\def\PYG@tc##1{\textcolor[rgb]{0.00,0.00,0.50}{##1}}}
\expandafter\def\csname PYG@tok@ni\endcsname{\let\PYG@bf=\textbf\def\PYG@tc##1{\textcolor[rgb]{0.84,0.33,0.22}{##1}}}
\expandafter\def\csname PYG@tok@nl\endcsname{\let\PYG@bf=\textbf\def\PYG@tc##1{\textcolor[rgb]{0.00,0.13,0.44}{##1}}}
\expandafter\def\csname PYG@tok@nn\endcsname{\let\PYG@bf=\textbf\def\PYG@tc##1{\textcolor[rgb]{0.05,0.52,0.71}{##1}}}
\expandafter\def\csname PYG@tok@no\endcsname{\def\PYG@tc##1{\textcolor[rgb]{0.38,0.68,0.84}{##1}}}
\expandafter\def\csname PYG@tok@na\endcsname{\def\PYG@tc##1{\textcolor[rgb]{0.25,0.44,0.63}{##1}}}
\expandafter\def\csname PYG@tok@nb\endcsname{\def\PYG@tc##1{\textcolor[rgb]{0.00,0.44,0.13}{##1}}}
\expandafter\def\csname PYG@tok@nc\endcsname{\let\PYG@bf=\textbf\def\PYG@tc##1{\textcolor[rgb]{0.05,0.52,0.71}{##1}}}
\expandafter\def\csname PYG@tok@nd\endcsname{\let\PYG@bf=\textbf\def\PYG@tc##1{\textcolor[rgb]{0.33,0.33,0.33}{##1}}}
\expandafter\def\csname PYG@tok@ne\endcsname{\def\PYG@tc##1{\textcolor[rgb]{0.00,0.44,0.13}{##1}}}
\expandafter\def\csname PYG@tok@nf\endcsname{\def\PYG@tc##1{\textcolor[rgb]{0.02,0.16,0.49}{##1}}}
\expandafter\def\csname PYG@tok@si\endcsname{\let\PYG@it=\textit\def\PYG@tc##1{\textcolor[rgb]{0.44,0.63,0.82}{##1}}}
\expandafter\def\csname PYG@tok@s2\endcsname{\def\PYG@tc##1{\textcolor[rgb]{0.25,0.44,0.63}{##1}}}
\expandafter\def\csname PYG@tok@nt\endcsname{\let\PYG@bf=\textbf\def\PYG@tc##1{\textcolor[rgb]{0.02,0.16,0.45}{##1}}}
\expandafter\def\csname PYG@tok@nv\endcsname{\def\PYG@tc##1{\textcolor[rgb]{0.73,0.38,0.84}{##1}}}
\expandafter\def\csname PYG@tok@s1\endcsname{\def\PYG@tc##1{\textcolor[rgb]{0.25,0.44,0.63}{##1}}}
\expandafter\def\csname PYG@tok@ch\endcsname{\let\PYG@it=\textit\def\PYG@tc##1{\textcolor[rgb]{0.25,0.50,0.56}{##1}}}
\expandafter\def\csname PYG@tok@m\endcsname{\def\PYG@tc##1{\textcolor[rgb]{0.13,0.50,0.31}{##1}}}
\expandafter\def\csname PYG@tok@gp\endcsname{\let\PYG@bf=\textbf\def\PYG@tc##1{\textcolor[rgb]{0.78,0.36,0.04}{##1}}}
\expandafter\def\csname PYG@tok@sh\endcsname{\def\PYG@tc##1{\textcolor[rgb]{0.25,0.44,0.63}{##1}}}
\expandafter\def\csname PYG@tok@ow\endcsname{\let\PYG@bf=\textbf\def\PYG@tc##1{\textcolor[rgb]{0.00,0.44,0.13}{##1}}}
\expandafter\def\csname PYG@tok@sx\endcsname{\def\PYG@tc##1{\textcolor[rgb]{0.78,0.36,0.04}{##1}}}
\expandafter\def\csname PYG@tok@bp\endcsname{\def\PYG@tc##1{\textcolor[rgb]{0.00,0.44,0.13}{##1}}}
\expandafter\def\csname PYG@tok@c1\endcsname{\let\PYG@it=\textit\def\PYG@tc##1{\textcolor[rgb]{0.25,0.50,0.56}{##1}}}
\expandafter\def\csname PYG@tok@o\endcsname{\def\PYG@tc##1{\textcolor[rgb]{0.40,0.40,0.40}{##1}}}
\expandafter\def\csname PYG@tok@kc\endcsname{\let\PYG@bf=\textbf\def\PYG@tc##1{\textcolor[rgb]{0.00,0.44,0.13}{##1}}}
\expandafter\def\csname PYG@tok@c\endcsname{\let\PYG@it=\textit\def\PYG@tc##1{\textcolor[rgb]{0.25,0.50,0.56}{##1}}}
\expandafter\def\csname PYG@tok@mf\endcsname{\def\PYG@tc##1{\textcolor[rgb]{0.13,0.50,0.31}{##1}}}
\expandafter\def\csname PYG@tok@err\endcsname{\def\PYG@bc##1{\setlength{\fboxsep}{0pt}\fcolorbox[rgb]{1.00,0.00,0.00}{1,1,1}{\strut ##1}}}
\expandafter\def\csname PYG@tok@mb\endcsname{\def\PYG@tc##1{\textcolor[rgb]{0.13,0.50,0.31}{##1}}}
\expandafter\def\csname PYG@tok@ss\endcsname{\def\PYG@tc##1{\textcolor[rgb]{0.32,0.47,0.09}{##1}}}
\expandafter\def\csname PYG@tok@sr\endcsname{\def\PYG@tc##1{\textcolor[rgb]{0.14,0.33,0.53}{##1}}}
\expandafter\def\csname PYG@tok@mo\endcsname{\def\PYG@tc##1{\textcolor[rgb]{0.13,0.50,0.31}{##1}}}
\expandafter\def\csname PYG@tok@kd\endcsname{\let\PYG@bf=\textbf\def\PYG@tc##1{\textcolor[rgb]{0.00,0.44,0.13}{##1}}}
\expandafter\def\csname PYG@tok@mi\endcsname{\def\PYG@tc##1{\textcolor[rgb]{0.13,0.50,0.31}{##1}}}
\expandafter\def\csname PYG@tok@kn\endcsname{\let\PYG@bf=\textbf\def\PYG@tc##1{\textcolor[rgb]{0.00,0.44,0.13}{##1}}}
\expandafter\def\csname PYG@tok@cpf\endcsname{\let\PYG@it=\textit\def\PYG@tc##1{\textcolor[rgb]{0.25,0.50,0.56}{##1}}}
\expandafter\def\csname PYG@tok@kr\endcsname{\let\PYG@bf=\textbf\def\PYG@tc##1{\textcolor[rgb]{0.00,0.44,0.13}{##1}}}
\expandafter\def\csname PYG@tok@s\endcsname{\def\PYG@tc##1{\textcolor[rgb]{0.25,0.44,0.63}{##1}}}
\expandafter\def\csname PYG@tok@kp\endcsname{\def\PYG@tc##1{\textcolor[rgb]{0.00,0.44,0.13}{##1}}}
\expandafter\def\csname PYG@tok@w\endcsname{\def\PYG@tc##1{\textcolor[rgb]{0.73,0.73,0.73}{##1}}}
\expandafter\def\csname PYG@tok@kt\endcsname{\def\PYG@tc##1{\textcolor[rgb]{0.56,0.13,0.00}{##1}}}
\expandafter\def\csname PYG@tok@sc\endcsname{\def\PYG@tc##1{\textcolor[rgb]{0.25,0.44,0.63}{##1}}}
\expandafter\def\csname PYG@tok@sb\endcsname{\def\PYG@tc##1{\textcolor[rgb]{0.25,0.44,0.63}{##1}}}
\expandafter\def\csname PYG@tok@k\endcsname{\let\PYG@bf=\textbf\def\PYG@tc##1{\textcolor[rgb]{0.00,0.44,0.13}{##1}}}
\expandafter\def\csname PYG@tok@se\endcsname{\let\PYG@bf=\textbf\def\PYG@tc##1{\textcolor[rgb]{0.25,0.44,0.63}{##1}}}
\expandafter\def\csname PYG@tok@sd\endcsname{\let\PYG@it=\textit\def\PYG@tc##1{\textcolor[rgb]{0.25,0.44,0.63}{##1}}}

\def\PYGZbs{\char`\\}
\def\PYGZus{\char`\_}
\def\PYGZob{\char`\{}
\def\PYGZcb{\char`\}}
\def\PYGZca{\char`\^}
\def\PYGZam{\char`\&}
\def\PYGZlt{\char`\<}
\def\PYGZgt{\char`\>}
\def\PYGZsh{\char`\#}
\def\PYGZpc{\char`\%}
\def\PYGZdl{\char`\$}
\def\PYGZhy{\char`\-}
\def\PYGZsq{\char`\'}
\def\PYGZdq{\char`\"}
\def\PYGZti{\char`\~}
% for compatibility with earlier versions
\def\PYGZat{@}
\def\PYGZlb{[}
\def\PYGZrb{]}
\makeatother

\renewcommand\PYGZsq{\textquotesingle}

\begin{document}

\maketitle
\tableofcontents
\phantomsection\label{bopcat::doc}


Contents:


\chapter{Introduction}
\label{intro:introduction}\label{intro:welcome-to-bopcat-s-documentation}\label{intro::doc}

\section{About}
\label{intro:about}
The \textbf{BOPcat} package is a software written in Python to generate
new tight-binding or bond-order potentials or optimize existing models using
the \textbf{BOPfox} code. The parameters of the models are determined by reproducing
various target properties including energies, forces, stresses, defect formation energies,
elastic constants, etc. The structures and their properties are taken from a DFT database
but can also include experiments or other data sources.
It employs the optimization libraries of \textbf{Scipy} but external optimization
modules can also be used.
The structures and their properties are handled using the \textbf{ASE} Atoms object.
These are read from a text file which are generated from an external
database.


\section{Installation}
\label{intro:installation}
Prior to using the code, BOPfox should be compiled and the
path to the executable should be included in \code{\$PATH}. As some BOPfox source files
are read to generate some variables, it is recommended to keep the executable in the
\code{bopfox/src} folder. BOPcat utilizes a number of python modules
such as numpy (\url{http://www.numpy.org/}), ASE (\url{https://wiki.fysik.dtu.dk/ase/}), Scipy (\url{http://www.scipy.org/}),
matplotlib (\url{http://www.matplotlib.org/}) and pyspglib (\url{https://atztogo.github.io/spglib/}). BOPcat depends
on the the BOPfox-ASE interface to calculate the properties of the structure which is not
yet included in the original ASE module. The relevant files are included in the
BOPcat source files. The files \code{bopio.py} and \code{bopcal.py} should be
copied to the \code{ase/io} and \code{ase/calculators} folders respectively. Both files should be renamed
\code{bopfox.py}. In order to know the path to your ase libraries, simpy execute the following in
python:

\begin{Verbatim}[commandchars=\\\{\}]
\PYG{k+kn}{import} \PYG{n+nn}{ase}
\PYG{n}{ase}\PYG{o}{.}\PYG{n}{\PYGZus{}\PYGZus{}file\PYGZus{}\PYGZus{}}
\end{Verbatim}

In case you do not have permission to make changes to the ase folder, it is necessary to
install a local version of ASE (\url{https://wiki.fysik.dtu.dk/ase/}). A more straightforward
way is to install \textbf{Anaconda} (\url{https://www.continuum.io/downloads}). The latter is recommended
as this will also update and consolidate all your python modules in a local directory.

\begin{notice}{note}{Note:}
It may be necessary to restart your computer.
\end{notice}

To make sure that you have set the correct path to \code{ase}, execute the following in python:

\begin{Verbatim}[commandchars=\\\{\}]
\PYG{k+kn}{from} \PYG{n+nn}{ase}\PYG{n+nn}{.}\PYG{n+nn}{calculators} \PYG{k}{import} \PYG{n}{bopfox} \PYG{k}{as} \PYG{n}{bopcal}
\PYG{k+kn}{from} \PYG{n+nn}{ase}\PYG{n+nn}{.}\PYG{n+nn}{io} \PYG{k}{import} \PYG{n}{bopfox} \PYG{k}{as} \PYG{n}{bopio}
\end{Verbatim}

To install \code{BOPcat}, run the installation script

\begin{Verbatim}[commandchars=\\\{\}]
\PYG{g+go}{python setup.py install}
\end{Verbatim}

Alternatively, one simply specify the path to the BOPcat source files, i.e. you should
append the following to your \code{.bashrc}:
\begin{quote}

\code{export PYTHONPATH=\textless{}path to bopcat\textgreater{}:/bopcat:\$PYTHONPATH}
\code{export PYTHONPATH=\textless{}path to bopcat\textgreater{}:\$PYTHONPATH}
\end{quote}

These make it possible to execute:

\begin{Verbatim}[commandchars=\\\{\}]
\PYG{k+kn}{from} \PYG{n+nn}{bopcat} \PYG{k}{import} \PYG{n}{variables}
\PYG{k+kn}{import} \PYG{n+nn}{variables}
\end{Verbatim}

To test if the required paths are set correctly, execute

\begin{Verbatim}[commandchars=\\\{\}]
\PYG{g+go}{python test\PYGZus{}install.py path}
\end{Verbatim}


\section{Examples}
\label{intro:examples}
The examples to test the basic functionalities of the code are found in \code{examples/}.
To get started, run the examples in the following order:
\begin{enumerate}
\item {} \begin{description}
\item[{\code{ASE}}] \leavevmode
The usage of the BOPfox ASE interface is illustrated.
See \href{https://dev.icams.rub.de/projects/projects/smap3d/repository/changes/examples/ASE/example1.py}{example1.py} and
\href{https://dev.icams.rub.de/projects/projects/smap3d/repository/changes/examples/ASE/example2.py}{example2.py}

\end{description}

\item {} \begin{description}
\item[{\code{strucscan}}] \leavevmode
BOPcat reference data are constructed from strucscan.
See \href{https://dev.icams.rub.de/projects/projects/smap3d/repository/changes/examples/strucscan/example.py}{example.py}

\end{description}

\item {} \begin{description}
\item[{\code{optimize\_Fe-Madsen-2011}}] \leavevmode
An existing Fe model is optimized.
See \href{https://dev.icams.rub.de/projects/projects/smap3d/repository/changes/examples/optimize\_Fe-Madsen-2011/input.py}{input2.py} and
\href{https://dev.icams.rub.de/projects/projects/smap3d/repository/changes/examples/optimize\_Fe-Madsen-2011/main.py}{main2.py}

\end{description}

\item {} \begin{description}
\item[{\code{optimize\_Re-Cak-2014}}] \leavevmode
An existing W model is optimized for Re.
See \href{https://dev.icams.rub.de/projects/projects/smap3d/repository/changes/examples/optimize\_Re-Cak-2014/input.py}{input3.py} and
\href{https://dev.icams.rub.de/projects/projects/smap3d/repository/changes/examples/optimize\_Re-Cak-2014/main.py}{main3.py}

\end{description}

\item {} \begin{description}
\item[{\code{construct\_Fe}}] \leavevmode
A new Fe model is constructed.
See \href{https://dev.icams.rub.de/projects/projects/smap3d/repository/changes/examples/construct\_Fe/input.py}{input4.py} and
\href{https://dev.icams.rub.de/projects/projects/smap3d/repository/changes/examples/construct\_Fe/main.py}{main4.py}

\end{description}

\item {} \begin{description}
\item[{\code{construct\_FeNb}}] \leavevmode
A new FeNb model is constructed.
See \href{https://dev.icams.rub.de/projects/projects/smap3d/repository/changes/examples/construct\_FeNb/input.py}{input5.py} and
\href{https://dev.icams.rub.de/projects/projects/smap3d/repository/changes/examples/construct\_FeNb/main.py}{main5.py}

\end{description}

\item {} \begin{description}
\item[{\code{test\_Fe\_Madsen-2011}}] \leavevmode
BOPcat utilities are illustrated.
See \href{https://dev.icams.rub.de/projects/projects/smap3d/repository/changes/examples/test\_Fe\_Madsen-2011/input.py}{input6.py} and
\href{https://dev.icams.rub.de/projects/projects/smap3d/repository/changes/examples/test\_Fe\_Madsen-2011/main.py}{main6.py}

\end{description}

\end{enumerate}

BOPcat is a collection of tools for the optimizing models. It is necessary for
the user to write a script to specify the procedure. In addition, the input controls
should also be provided which are handled by the \code{cat\_controls} object.
Assuming that the main script is \code{main.py} and the input file is \code{input.py}, the scripts should
be executed as

\begin{Verbatim}[commandchars=\\\{\}]
\PYG{g+go}{python main.py input.py}
\end{Verbatim}

The basic form of the script is as follows:
\begin{enumerate}
\item {} 
Execute and initialize the input controls. These are then attributes of the \code{cat\_controls} object:

\begin{Verbatim}[commandchars=\\\{\}]
\PYG{n}{execfile}\PYG{p}{(}\PYG{n}{sys}\PYG{o}{.}\PYG{n}{argv}\PYG{p}{[}\PYG{o}{\PYGZhy{}}\PYG{l+m+mi}{1}\PYG{p}{]}\PYG{p}{)}
\PYG{n}{cat\PYGZus{}contols}\PYG{o}{.}\PYG{n}{initialize}\PYG{p}{(}\PYG{p}{)}
\end{Verbatim}

\item {} 
Generate reference data. The \code{cat\_data} object essentially reads the text file of structures and their properties (see {\hyperref[refdata:refdata]{\crossref{\DUrole{std,std-ref}{Reference structures database}}}}):

\begin{Verbatim}[commandchars=\\\{\}]
\PYG{n}{cat\PYGZus{}data} \PYG{o}{=} \PYG{n}{CATData}\PYG{p}{(}\PYG{n}{controls}\PYG{o}{=}\PYG{n}{cat\PYGZus{}controls}\PYG{p}{)}
\end{Verbatim}

\item {} 
Generate or read initial model. The \code{cat\_param} object can also be used to store the resulting models at each level of optimization:

\begin{Verbatim}[commandchars=\\\{\}]
\PYG{n}{cat\PYGZus{}param} \PYG{o}{=} \PYG{n}{CATParam}\PYG{p}{(}\PYG{n}{controls}\PYG{o}{=}\PYG{n}{cat\PYGZus{}controls}\PYG{p}{,}\PYG{n}{data}\PYG{o}{=}\PYG{n}{cat\PYGZus{}data}\PYG{p}{)}
\end{Verbatim}

\item {} 
Set up the calculator. To set up the calculator, one provides the input controls and a model, in the following, we use the last model (\code{models{[}-1{]}}) saved in \code{cat\_param}

\begin{Verbatim}[commandchars=\\\{\}]
\PYG{n}{cat\PYGZus{}calc} \PYG{o}{=} \PYG{n}{CATCalc}\PYG{p}{(}\PYG{n}{controls}\PYG{o}{=}\PYG{n}{cat\PYGZus{}controls}\PYG{p}{,}\PYG{n}{model}\PYG{o}{=}\PYG{n}{cat\PYGZus{}param}\PYG{o}{.}\PYG{n}{models}\PYG{p}{[}\PYG{o}{\PYGZhy{}}\PYG{l+m+mi}{1}\PYG{p}{]}\PYG{p}{)}
\end{Verbatim}

\item {} 
To proceed with the optimization, one needs to determine which structures are included in the target set. These are then assigned to the calculator:

\begin{Verbatim}[commandchars=\\\{\}]
\PYG{n}{ref\PYGZus{}atoms} \PYG{o}{=} \PYG{n}{cat\PYGZus{}data}\PYG{o}{.}\PYG{n}{get\PYGZus{}ref\PYGZus{}atoms}\PYG{p}{(}\PYG{n}{structures}\PYG{o}{=}\PYG{p}{[}\PYG{l+s+s1}{\PYGZsq{}}\PYG{l+s+s1}{Fe/229/0/1/*}\PYG{l+s+s1}{\PYGZsq{}}\PYG{p}{]}\PYG{p}{,}\PYG{n}{quantities}\PYG{o}{=}\PYG{p}{[}\PYG{l+s+s1}{\PYGZsq{}}\PYG{l+s+s1}{energy}\PYG{l+s+s1}{\PYGZsq{}}\PYG{p}{]}\PYG{p}{)}
\PYG{n}{cat\PYGZus{}calc}\PYG{o}{.}\PYG{n}{set\PYGZus{}atoms}\PYG{p}{(}\PYG{n}{ref\PYGZus{}atoms}\PYG{p}{)}
\PYG{n}{ref\PYGZus{}data} \PYG{o}{=} \PYG{n}{cat\PYGZus{}data}\PYG{o}{.}\PYG{n}{get\PYGZus{}ref\PYGZus{}data}\PYG{p}{(}\PYG{p}{)}
\end{Verbatim}

\item {} 
The optimization kernel is also necessary which takes in the calculator, the array of reference data, the constraints on variables, the weights and the input controls:

\begin{Verbatim}[commandchars=\\\{\}]
\PYG{n}{var}  \PYG{o}{=} \PYG{p}{[}\PYG{p}{\PYGZob{}}\PYG{l+s+s2}{\PYGZdq{}}\PYG{l+s+s2}{bond}\PYG{l+s+s2}{\PYGZdq{}}\PYG{p}{:}\PYG{p}{[}\PYG{l+s+s2}{\PYGZdq{}}\PYG{l+s+s2}{Fe}\PYG{l+s+s2}{\PYGZdq{}}\PYG{p}{,}\PYG{l+s+s2}{\PYGZdq{}}\PYG{l+s+s2}{Fe}\PYG{l+s+s2}{\PYGZdq{}}\PYG{p}{]}\PYG{p}{,}\PYG{l+s+s1}{\PYGZsq{}}\PYG{l+s+s1}{atom}\PYG{l+s+s1}{\PYGZsq{}}\PYG{p}{:}\PYG{l+s+s1}{\PYGZsq{}}\PYG{l+s+s1}{Fe}\PYG{l+s+s1}{\PYGZsq{}}\PYG{p}{,}\PYG{l+s+s1}{\PYGZsq{}}\PYG{l+s+s1}{onsitelevels}\PYG{l+s+s1}{\PYGZsq{}}\PYG{p}{:}\PYG{p}{[}\PYG{k+kc}{True}\PYG{p}{]}\PYG{p}{\PYGZcb{}}\PYG{p}{]}
\PYG{n}{optfunc} \PYG{o}{=} \PYG{n}{CATkernel}\PYG{p}{(}\PYG{n}{calc}\PYG{o}{=}\PYG{n}{cat\PYGZus{}calc}\PYG{p}{,}\PYG{n}{ref\PYGZus{}data}\PYG{o}{=}\PYG{n}{ref\PYGZus{}data}\PYG{p}{,}\PYG{n}{variables}\PYG{o}{=}\PYG{n}{var}\PYG{p}{,}\PYG{n}{log}\PYG{o}{=}\PYG{l+s+s1}{\PYGZsq{}}\PYG{l+s+s1}{log.cat}\PYG{l+s+s1}{\PYGZsq{}}
                   \PYG{p}{,}\PYG{n}{controls}\PYG{o}{=}\PYG{n}{cat\PYGZus{}controls}\PYG{p}{)}
\end{Verbatim}

\item {} 
The optimization is run and the resulting model is saved in \code{cat\_param}:

\begin{Verbatim}[commandchars=\\\{\}]
\PYG{n}{optfunc}\PYG{o}{.}\PYG{n}{optimize}\PYG{p}{(}\PYG{p}{)}
\PYG{n}{new\PYGZus{}model} \PYG{o}{=} \PYG{n}{optfunc}\PYG{o}{.}\PYG{n}{get\PYGZus{}optimized\PYGZus{}model}\PYG{p}{(}\PYG{p}{)}
\PYG{n}{cat\PYGZus{}param}\PYG{o}{.}\PYG{n}{models}\PYG{o}{.}\PYG{n}{append}\PYG{p}{(}\PYG{n}{new\PYGZus{}model}\PYG{p}{)}
\end{Verbatim}

\end{enumerate}

This can be done iteratively for different sets of target structures, constraints, starting parameters or even a completely different functional form.


\chapter{Reference structures database}
\label{refdata:refdata}\label{refdata::doc}\label{refdata:reference-structures-database}
BOPcat requires a list of structures each is an \code{ase.Atoms} object.
These are extracted from a generic database which takes in general identifiers
depending on the data type. Currently, only \code{dft} data are handled in
but can be extended to include data from experiments or
other calculation schemes. The \code{Atoms} also contain the calculation
parameters, details of the crystal structure and various properties that are
required for fitting. The data are extracted from the database once and are written in
a readable text format for future use.
\phantomsection\label{refdata:general-dft-identifiers}\begin{quote}

\textbf{General DFT identifiers}

\begin{longtable}{|l|l|l|l|l|}
\hline
\textsf{\relax 
key
} & \textsf{\relax 
val
} & \textsf{\relax 
description
} & \textsf{\relax 
val
} & \textsf{\relax 
description
}\\
\hline\endfirsthead

\multicolumn{5}{c}%
{{\tablecontinued{\tablename\ \thetable{} -- continued from previous page}}} \\
\hline
\textsf{\relax 
key
} & \textsf{\relax 
val
} & \textsf{\relax 
description
} & \textsf{\relax 
val
} & \textsf{\relax 
description
}\\
\hline\endhead

\hline \multicolumn{5}{|r|}{{\tablecontinued{Continued on next page}}} \\ \hline
\endfoot

\endlastfoot


code
 & 
0
 & 
abinit
 & 
100
 & 
crystal
\\
\hline & 
1
 & 
castep
 & 
101
 & 
fhi-aims
\\
\hline & 
2
 & 
dacapo
 & 
102
 & 
gaussian
\\
\hline & 
3
 & 
gpaw
 & 
103
 & 
octopus
\\
\hline & 
4
 & 
qbox
 & 
104
 & 
siesta
\\
\hline & 
5
 & 
quantum-espresso
 & 
105
 & 
turbomole
\\
\hline & 
6
 & 
sphinx
 &  & \\
\hline & 
7
 & 
vasp
 &  & \\
\hline & 
200
 & 
exciting
 & 
999
 & 
unknown
\\
\hline & 
201
 & 
fleur
 &  & \\
\hline & 
202
 & 
wien2k
 &  & \\
\hline
basis\_set
 & 
0
 & 
plane waves
 &  & \\
\hline & 
1
 & 
atomic orbitals
 &  & \\
\hline & 
2
 & 
gaussians
 &  & \\
\hline & 
9
 & 
unknown
 &  & \\
\hline
xc\_functional
 & 
0
 & 
hf ab-initio
 & 
100
 & 
pw lda
\\
\hline & 
1
 & 
sx ab-initio
 & 
101
 & 
pz lda
\\
\hline &  &  & 
102
 & 
vwn lda
\\
\hline & 
200
 & 
blyp gga
 & 
300
 & 
tpss meta
\\
\hline & 
201
 & 
pbe gga
 &  & \\
\hline & 
202
 & 
pw91 gga
 &  & \\
\hline & 
203
 & 
rpbe gga
 &  & \\
\hline & 
204
 & 
pbesol gga
 &  & \\
\hline & 
205
 & 
wc gga
 &  & \\
\hline & 
206
 & 
lm gga
 &  & \\
\hline & 
400
 & 
b3lyp hybrid
 & 
999
 & 
unknown
\\
\hline & 
401
 & 
hse hybrid
 &  & \\
\hline & 
402
 & 
pbe0 hybrid
 &  & \\
\hline & 
403
 & 
tpssh hybrid
 &  & \\
\hline
pseudopotential
 & 
0
 & 
all electron
 & 
10
 & 
martins-troullier nc
\\
\hline &  &  & 
11
 & 
bachelet-hamman-schlueter nc
\\
\hline &  &  & 
12
 & 
morrison-bylander-kleinman nc
\\
\hline & 
20
 & 
vanderbilt us
 & 
30
 & 
kresse-joubert paw
\\
\hline & 
21
 & 
rappe-rabe-kaxiras-joannopoulos us
 & 
99
 & 
unknown
\\
\hline\end{longtable}

\end{quote}

For \code{dft} reference data, four general identifiers namely \code{code}, \code{xc\_functional},
\code{pseudopotential} and \code{basis set} are passed to the database. Each of these
identifiers have a corresponding integer value for the purpose of
classification (see {\hyperref[refdata:general\string-dft\string-identifiers]{\crossref{General DFT identifiers}}}). The DFT codes are arranged
in the following manner. The first set use mainly plawe wave basis while those
from 100-199 use local atomic
orbital basis while the third are all-electron DFT codes. The classification for the
other identifiers are are also evident from {\hyperref[refdata:general\string-dft\string-identifiers]{\crossref{General DFT identifiers}}}. The database
returns a list of \code{Atoms} that meet these criteria for the general identifiers.
The database is expected to also supply supplementary identifiers listed in {\hyperref[refdata:supplementary\string-dft\string-identifiers]{\crossref{Supplementary DFT identifiers}}}
in \code{Atoms.info}. The information are used in BOPcat to filter the data. To extend
the supplementary identifiers, one simply add keywords in \code{variables.data\_keys()}
\phantomsection\label{refdata:supplementary-dft-identifiers}\begin{quote}

\textbf{Supplementary DFT identifiers}

\begin{tabulary}{\linewidth}{|L|L|L|}
\hline
\textsf{\relax 
key
} & \textsf{\relax 
val
} & \textsf{\relax 
description
}\\
\hline
hubbard
 & 
0.0
 & 
U value
\\
\hline
lr\_correction
 & 
0/1
 & 
long-range correction
\\
\hline
relativistic
 & 
0/1
 & 
relativistic
\\
\hline
valency
 & 
3pd7s1
 & 
valence states
\\
\hline
nlc\_correction
 & 
0/1
 & 
non-linear core correction
\\
\hline
encut
 & 
0.0
 & 
cut-off energy
\\
\hline
deltak
 & 
0.0
 & 
k-points(1/Angstroem)
\\
\hline
encut\_ok
 & 
0/1
 & 
converged wrt encut
\\
\hline
deltak\_ok
 & 
0/1
 & 
converged wrt deltak
\\
\hline
author
 & 
Pedro5
 & 
author ID
\\
\hline\end{tabulary}

\end{quote}

The database should also return identifiers for the structure. These
include the \code{stoichiometry}, \code{space group number}, \code{sytem type},
\code{calculation type} and \code{calculation order}. The \code{system\_type} can refer
to any of the following: \code{bulk}, \code{cluster}, \code{defect}, \code{surface} or \code{interface}.
An integer value is also assigned to the system types, i.e.

\code{0-bulk 1-cluster 2-defect 3-surface 4-interface}

\begin{notice}{note}{Todo}

Extend system types.
\end{notice}

The nature of \code{calculation\_type} and \code{calculation\_order} vary depending on \code{system\_type}. These
are listed in {\hyperref[refdata:structure\string-identifiers]{\crossref{Structure identifiers}}}.
\phantomsection\label{refdata:structure-identifiers}\begin{quote}

\textbf{Structure identifiers}

\begin{tabulary}{\linewidth}{|L|L|L|L|}
\hline
\textsf{\relax 
calc\_type
} & \textsf{\relax 
description
} & \textsf{\relax 
calc\_order
} & \textsf{\relax 
description
}\\
\hline \multicolumn{4}{|l|}{
\emph{bulk}
}\\
\hline
0
 & 
relaxation
 & 
N
 & 
0-unrelax, 1-relax ion, 2-relax vol, 3-relax all
\\
\hline
1
 & 
volume
 & 
0.0
 & 
volume per atom
\\
\hline
2
 & 
elastic
 & 
v.x
 & 
v-strain(Voigt), x-displacement
\\
\hline
3
 & 
phonon
 & 
Iv.x
 & 
I-atom index, v-strain(Voigt), x-displacement
\\
\hline
4
 & 
transformation
 & 
Td.x
 & 
T-transformation type, d.x-deformation factor
\\
\hline
99
 & 
unknown
 & 
99.99
 & 
unknown
\\
\hline\end{tabulary}

\end{quote}

\begin{notice}{note}{Todo}

Determine identifier for other system types.
\end{notice}

Each structure may contain an energy, forces,stresses, eigenvalues and orbital\_character,
vacancy energy and other properties. To extend this to include other properties, one
simply needs to add the key to the \code{variables.data\_keys()}.


\chapter{Classes and Modules}
\label{classes::doc}\label{classes:classes-and-modules}

\section{Input controls}
\label{classes:input-controls}\label{classes:module-catcontrols}\index{catcontrols (module)}\index{CATControls (class in catcontrols)}

\begin{fulllineitems}
\phantomsection\label{classes:catcontrols.CATControls}\pysiglinewithargsret{\strong{class }\code{catcontrols.}\bfcode{CATControls}}{\emph{**kwargs}}{}
Handles all input controls.

Parameters are categorized as either data-, opt- calculator-
and model-specific
\begin{quote}\begin{description}
\item[{Parameters}] \leavevmode\begin{itemize}
\item {} 
\emph{elements}: list
\begin{quote}

list of chemical symbols. Bond pairs will be generated from the
list.
\end{quote}

\item {} 
\emph{calculator\_settings}: dict
\begin{quote}

options to be passed to the calculator
\end{quote}

\item {} 
\emph{calculator}: str
\begin{quote}

name of the calculator
\end{quote}

\item {} 
\emph{calculator\_nproc}: int
\begin{quote}

number of parallel processes

\code{None}: will not parallelize
\end{quote}

\item {} 
\emph{data\_parameters}: dict
\begin{quote}

specifications for the data, e.g.:

\begin{Verbatim}[commandchars=\\\{\}]
\PYG{n}{data\PYGZus{}parameters} \PYG{o}{=} \PYG{p}{\PYGZob{}}\PYG{l+s+s1}{\PYGZsq{}}\PYG{l+s+s1}{xc\PYGZus{}functional}\PYG{l+s+s1}{\PYGZsq{}}\PYG{p}{:}\PYG{l+m+mi}{30}\PYG{p}{,} \PYG{l+s+s1}{\PYGZsq{}}\PYG{l+s+s1}{encut}\PYG{l+s+s1}{\PYGZsq{}}\PYG{p}{:}\PYG{l+m+mi}{400}\PYG{p}{\PYGZcb{}}
\end{Verbatim}
\end{quote}

\item {} 
\emph{data\_system\_parameters}: dict
\begin{quote}

specifications for the structures, e.g.:

\begin{Verbatim}[commandchars=\\\{\}]
\PYG{n}{data\PYGZus{}system\PYGZus{}parameters} \PYG{o}{=} \PYG{p}{\PYGZob{}}\PYG{l+s+s1}{\PYGZsq{}}\PYG{l+s+s1}{stoichiometry}\PYG{l+s+s1}{\PYGZsq{}}\PYG{p}{:}\PYG{p}{[}\PYG{l+s+s1}{\PYGZsq{}}\PYG{l+s+s1}{Fe2}\PYG{l+s+s1}{\PYGZsq{}}\PYG{p}{,}\PYG{l+s+s1}{\PYGZsq{}}\PYG{l+s+s1}{Fe}\PYG{l+s+s1}{\PYGZsq{}}\PYG{p}{]}\PYG{p}{\PYGZcb{}}
\end{Verbatim}
\end{quote}

\item {} 
\emph{data\_filename}: str
\begin{quote}

filename or path to file containing the structures and properties
\end{quote}

\item {} 
\emph{data\_free\_atom\_energies}: dict
\begin{quote}

dictionary of element:free atom energies

\code{dimer}: will generate free\_atom\_energies from dimers

\code{None}: will read from atomic\_properties in \code{variables}
\end{quote}

\item {} 
\emph{opt\_variables}: list
\begin{quote}

list of dictionary of contraints on the parameters:

\begin{Verbatim}[commandchars=\\\{\}]
\PYG{n}{variables} \PYG{o}{=} \PYG{p}{[}\PYG{p}{\PYGZob{}}\PYG{l+s+s1}{\PYGZsq{}}\PYG{l+s+s1}{bond}\PYG{l+s+s1}{\PYGZsq{}}\PYG{p}{:}\PYG{p}{[}\PYG{l+s+s1}{\PYGZsq{}}\PYG{l+s+s1}{Fe}\PYG{l+s+s1}{\PYGZsq{}}\PYG{p}{,}\PYG{l+s+s1}{\PYGZsq{}}\PYG{l+s+s1}{Fe}\PYG{l+s+s1}{\PYGZsq{}}\PYG{p}{]}\PYG{p}{,}\PYG{l+s+s1}{\PYGZsq{}}\PYG{l+s+s1}{rep1}\PYG{l+s+s1}{\PYGZsq{}}\PYG{p}{:}\PYG{p}{[}\PYG{k+kc}{True}\PYG{p}{,}\PYG{k+kc}{True}\PYG{p}{]}\PYG{p}{\PYGZcb{}}
            \PYG{p}{,}\PYG{p}{\PYGZob{}}\PYG{l+s+s1}{\PYGZsq{}}\PYG{l+s+s1}{bond}\PYG{l+s+s1}{\PYGZsq{}}\PYG{p}{:}\PYG{p}{[}\PYG{l+s+s1}{\PYGZsq{}}\PYG{l+s+s1}{Fe}\PYG{l+s+s1}{\PYGZsq{}}\PYG{p}{,}\PYG{l+s+s1}{\PYGZsq{}}\PYG{l+s+s1}{Nb}\PYG{l+s+s1}{\PYGZsq{}}\PYG{p}{]}\PYG{p}{,}\PYG{l+s+s1}{\PYGZsq{}}\PYG{l+s+s1}{rep2}\PYG{l+s+s1}{\PYGZsq{}}\PYG{p}{:}\PYG{p}{[}\PYG{k+kc}{True}\PYG{p}{,}\PYG{k+kc}{True}\PYG{p}{,}\PYG{k+kc}{True}\PYG{p}{]}\PYG{p}{\PYGZcb{}}
            \PYG{p}{,}\PYG{p}{\PYGZob{}}\PYG{l+s+s1}{\PYGZsq{}}\PYG{l+s+s1}{atom}\PYG{l+s+s1}{\PYGZsq{}}\PYG{p}{:}\PYG{p}{[}\PYG{l+s+s1}{\PYGZsq{}}\PYG{l+s+s1}{Fe}\PYG{l+s+s1}{\PYGZsq{}}\PYG{p}{]}\PYG{p}{,}\PYG{l+s+s1}{\PYGZsq{}}\PYG{l+s+s1}{valenceelectrons}\PYG{l+s+s1}{\PYGZsq{}}\PYG{p}{:}\PYG{p}{[}\PYG{k+kc}{True}\PYG{p}{]}\PYG{p}{\PYGZcb{}}\PYG{p}{]} 
\end{Verbatim}
\end{quote}

\item {} 
\emph{opt\_structures}: list
\begin{quote}

list of structures in target set:

\begin{Verbatim}[commandchars=\\\{\}]
\PYG{n}{structures} \PYG{o}{=} \PYG{p}{[}\PYG{l+s+s1}{\PYGZsq{}}\PYG{l+s+s1}{Fe*/*/0/1/*}\PYG{l+s+s1}{\PYGZsq{}}\PYG{p}{]}
\end{Verbatim}
\end{quote}

\item {} 
\emph{opt\_test\_structures}: list
\begin{quote}

list of structures in test set
\end{quote}

\item {} 
\emph{opt\_optimizer}: str
\begin{quote}

name of optimizer
\end{quote}

\item {} 
\emph{opt\_optimizer\_options}: dict
\begin{quote}

controls to be passed to the optimizer
\end{quote}

\item {} 
\emph{opt\_objective}: str
\begin{quote}

name of the objective function
\end{quote}

\item {} 
\emph{model}: str
\begin{quote}

name of the model
\end{quote}

\item {} 
\emph{model\_pathtomodels}: str
\begin{quote}

filename or path to file containing model
\end{quote}

\item {} 
\emph{model\_pathtobetas}: str
\begin{quote}

filename or path to file containing bond integrals
\end{quote}

\item {} 
\emph{model\_pathtoonsites}: str
\begin{quote}

filename or path to file containing onsites
\end{quote}

\item {} 
\emph{model\_functions}: dict
\begin{quote}

dictionary of keyword:bondpair:function or keyword:function, 
function is one of the instances in functions module. The latter
implies that all bond pairs will have the same functional form:

\begin{Verbatim}[commandchars=\\\{\}]
\PYG{k+kn}{import} \PYG{n+nn}{functions} \PYG{k}{as} \PYG{n+nn}{funcs}
\PYG{n}{functions} \PYG{o}{=} \PYG{p}{\PYGZob{}}\PYG{l+s+s1}{\PYGZsq{}}\PYG{l+s+s1}{ddsigma}\PYG{l+s+s1}{\PYGZsq{}}\PYG{p}{:}\PYG{p}{\PYGZob{}}\PYG{l+s+s1}{\PYGZsq{}}\PYG{l+s+s1}{Fe\PYGZhy{}Fe}\PYG{l+s+s1}{\PYGZsq{}}\PYG{p}{:}\PYG{n}{funcs}\PYG{o}{.}\PYG{n}{exponential}\PYG{p}{(}\PYG{p}{)}
                       \PYG{p}{,}\PYG{l+s+s1}{\PYGZsq{}}\PYG{l+s+s1}{Nb\PYGZhy{}Nb}\PYG{l+s+s1}{\PYGZsq{}}\PYG{p}{:}\PYG{n}{funcs}\PYG{o}{.}\PYG{n}{GSP}\PYG{p}{(}\PYG{p}{)}\PYG{p}{\PYGZcb{}}
                       \PYG{p}{,}\PYG{l+s+s1}{\PYGZsq{}}\PYG{l+s+s1}{rep1}\PYG{l+s+s1}{\PYGZsq{}}\PYG{p}{:}\PYG{n}{funcs}\PYG{o}{.}\PYG{n}{rep\PYGZus{}exponential}\PYG{p}{(}\PYG{p}{)}\PYG{p}{\PYGZcb{}}
\end{Verbatim}
\end{quote}

\item {} 
\emph{model\_valences}: dict
\begin{quote}

dictionary of element:valency:

\begin{Verbatim}[commandchars=\\\{\}]
\PYG{n}{valences} \PYG{o}{=} \PYG{p}{\PYGZob{}}\PYG{l+s+s1}{\PYGZsq{}}\PYG{l+s+s1}{Fe}\PYG{l+s+s1}{\PYGZsq{}}\PYG{p}{:}\PYG{l+s+s1}{\PYGZsq{}}\PYG{l+s+s1}{d}\PYG{l+s+s1}{\PYGZsq{}}\PYG{p}{,}\PYG{l+s+s1}{\PYGZsq{}}\PYG{l+s+s1}{Nb}\PYG{l+s+s1}{\PYGZsq{}}\PYG{p}{:}\PYG{l+s+s1}{\PYGZsq{}}\PYG{l+s+s1}{d}\PYG{l+s+s1}{\PYGZsq{}}\PYG{p}{\PYGZcb{}} 
\end{Verbatim}
\end{quote}

\item {} 
\emph{model\_valenceelectrons}: dict
\begin{quote}

dictionary of element:number of valence electrons:

\begin{Verbatim}[commandchars=\\\{\}]
\PYG{n}{valenceelectrons} \PYG{o}{=} \PYG{p}{\PYGZob{}}\PYG{l+s+s1}{\PYGZsq{}}\PYG{l+s+s1}{Fe}\PYG{l+s+s1}{\PYGZsq{}}\PYG{p}{:}\PYG{l+m+mf}{7.1}\PYG{p}{,}\PYG{l+s+s1}{\PYGZsq{}}\PYG{l+s+s1}{Nb}\PYG{l+s+s1}{\PYGZsq{}}\PYG{p}{:}\PYG{l+m+mf}{4.0}\PYG{p}{\PYGZcb{}} 
\end{Verbatim}

\code{None}: will read valenceelectrons from 
\code{variables.atomic\_properties()}
\end{quote}

\item {} 
\emph{model\_orthogonal}: bool
\begin{quote}

orthogonal or non-orthogonal model
\end{quote}

\item {} 
\emph{model\_betafitstruc}: str
\begin{quote}

structure from which the betas were derived (see format of betas 
files)
\end{quote}

\item {} 
\emph{model\_betatype}: str
\begin{quote}

method used in bond integral generation
\end{quote}

\item {} 
\emph{model\_betabasis}: str
\begin{quote}

basis used in bond integral generation
\end{quote}

\item {} 
\emph{model\_cutoff}: dict
\begin{quote}

dictionary of cutoff keyword:bondpair:value or keyword:value.

The latter implies that all bond pairs will have the same value:

\begin{Verbatim}[commandchars=\\\{\}]
\PYG{n}{cutoff} \PYG{o}{=} \PYG{p}{\PYGZob{}}\PYG{l+s+s1}{\PYGZsq{}}\PYG{l+s+s1}{rcut}\PYG{l+s+s1}{\PYGZsq{}} \PYG{p}{:} \PYG{p}{\PYGZob{}}\PYG{l+s+s1}{\PYGZsq{}}\PYG{l+s+s1}{Nb\PYGZhy{}Nb}\PYG{l+s+s1}{\PYGZsq{}}\PYG{p}{:}\PYG{l+m+mf}{4.5}\PYG{p}{,}\PYG{l+s+s1}{\PYGZsq{}}\PYG{l+s+s1}{Fe\PYGZhy{}Fe}\PYG{l+s+s1}{\PYGZsq{}}\PYG{p}{:}\PYG{l+m+mf}{4.5}\PYG{p}{,}\PYG{l+s+s1}{\PYGZsq{}}\PYG{l+s+s1}{Fe\PYGZhy{}Nb}\PYG{l+s+s1}{\PYGZsq{}}\PYG{p}{:}\PYG{l+m+mf}{4.5}\PYG{p}{\PYGZcb{}}
         \PYG{p}{,}\PYG{l+s+s1}{\PYGZsq{}}\PYG{l+s+s1}{dcut}\PYG{l+s+s1}{\PYGZsq{}} \PYG{p}{:}\PYG{l+m+mf}{0.5}\PYG{p}{\PYGZcb{}}
\end{Verbatim}
\end{quote}

\item {} 
\emph{verbose}: int
\begin{quote}

prints details at different levels
\end{quote}

\end{itemize}

\end{description}\end{quote}
\index{check\_input() (catcontrols.CATControls method)}

\begin{fulllineitems}
\phantomsection\label{classes:catcontrols.CATControls.check_input}\pysiglinewithargsret{\bfcode{check\_input}}{}{}
Checks the consistency of input data type.

\end{fulllineitems}

\index{convert\_params() (catcontrols.CATControls method)}

\begin{fulllineitems}
\phantomsection\label{classes:catcontrols.CATControls.convert_params}\pysiglinewithargsret{\bfcode{convert\_params}}{}{}
Convert data and system parameters into integers.

\end{fulllineitems}

\index{gen\_functions() (catcontrols.CATControls method)}

\begin{fulllineitems}
\phantomsection\label{classes:catcontrols.CATControls.gen_functions}\pysiglinewithargsret{\bfcode{gen\_functions}}{}{}
Set default functions. Returns dictonary of key:functions

\end{fulllineitems}

\index{initialize() (catcontrols.CATControls method)}

\begin{fulllineitems}
\phantomsection\label{classes:catcontrols.CATControls.initialize}\pysiglinewithargsret{\bfcode{initialize}}{}{}
Initializes input controls.

Calls {\hyperref[classes:catcontrols.CATControls.check_input]{\crossref{\code{check\_input()}}}}, {\hyperref[classes:catcontrols.CATControls.make_lower]{\crossref{\code{make\_lower()}}}}, \code{make\_pairs()},
{\hyperref[classes:catcontrols.CATControls.convert_params]{\crossref{\code{convert\_params()}}}}

\end{fulllineitems}

\index{make\_lower() (catcontrols.CATControls method)}

\begin{fulllineitems}
\phantomsection\label{classes:catcontrols.CATControls.make_lower}\pysiglinewithargsret{\bfcode{make\_lower}}{}{}
Change case of all strings in calculator\_settings.

\end{fulllineitems}


\end{fulllineitems}



\section{Reference data}
\label{classes:module-catdata}\label{classes:reference-data}\index{catdata (module)}\index{CATData (class in catdata)}

\begin{fulllineitems}
\phantomsection\label{classes:catdata.CATData}\pysiglinewithargsret{\strong{class }\code{catdata.}\bfcode{CATData}}{\emph{**kwargs}}{}
Defines a set of reference data for parametrization.

It stores the structures and their properties as ASE Atoms objects.

Extended properties used in the parametrization such as eigenvalues,
formation energies and elastic constants are stored in the info dictionary.

\begin{notice}{note}{Todo}

extend ASE Atoms object
\end{notice}

The main functionality of this object is to query structures for 
parameterization.
\begin{quote}\begin{description}
\item[{Parameters}] \leavevmode\begin{itemize}
\item {} 
\emph{controls}: instance of CATControls
\begin{quote}

CATControls object to initialize parameters
\end{quote}

\item {} 
\emph{atoms}: list of ASE Atoms instances
\begin{quote}

None: will read from file.
\end{quote}

\item {} 
\emph{filename}: str
\begin{quote}

file containing the structures and their properties, can also be 
path to file.
\end{quote}

\item {} 
\emph{dataparams}: dict
\begin{quote}

dictionary of data specifications used to filter structures.

sample keywords: \titleref{deltak}, \titleref{encut}
\end{quote}

\item {} 
\emph{sysparams}: dict
\begin{quote}

dictionary of system specifications used to filter structures.

sample keywords: \titleref{calculation\_type}, \titleref{spin}
\end{quote}

\item {} 
\emph{elements}: list
\begin{quote}

list of chemical symbols
\end{quote}

\item {} 
\emph{free\_atom\_energies}: dict
\begin{quote}

dictionary of element:free atom energies.

\code{dimer}: will generate free\_atom\_energies from dimers.

\code{None}: will read from atomic\_properties in \code{variables}.
\end{quote}

\item {} 
\emph{verbose}: int
\begin{quote}

controls verbosity e.g.

prints out structures if verbose \textgreater{} 1
\end{quote}

\end{itemize}

\end{description}\end{quote}
\index{get\_atoms() (catdata.CATData method)}

\begin{fulllineitems}
\phantomsection\label{classes:catdata.CATData.get_atoms}\pysiglinewithargsret{\bfcode{get\_atoms}}{}{}
Returns list of all ASE atoms objects of all structures. 
Builds them from file if not initialized.

\end{fulllineitems}

\index{get\_atoms\_info() (catdata.CATData method)}

\begin{fulllineitems}
\phantomsection\label{classes:catdata.CATData.get_atoms_info}\pysiglinewithargsret{\bfcode{get\_atoms\_info}}{\emph{key}}{}
Returns the \titleref{info{[}key{]}} values of all atoms
\begin{quote}\begin{description}
\item[{Parameters}] \leavevmode\begin{itemize}
\item {} 
\emph{key}: str
\begin{quote}

key of entry in info dictionary
\end{quote}

\end{itemize}

\end{description}\end{quote}

\end{fulllineitems}

\index{get\_equilibrium\_distance() (catdata.CATData method)}

\begin{fulllineitems}
\phantomsection\label{classes:catdata.CATData.get_equilibrium_distance}\pysiglinewithargsret{\bfcode{get\_equilibrium\_distance}}{\emph{structure}}{}
Returns the smallest bond length of the lowest energy structure
in the group
\begin{quote}\begin{description}
\item[{Parameters}] \leavevmode\begin{itemize}
\item {} 
\emph{structure}: str
\begin{quote}

strucname,  e.g. \titleref{`dimer'}

system\_ID, e.g. \titleref{`Fe/229/0/1/*'} : all bcc-Fe E-V structures
\end{quote}

\end{itemize}

\end{description}\end{quote}

\end{fulllineitems}

\index{get\_free\_atom\_energies() (catdata.CATData method)}

\begin{fulllineitems}
\phantomsection\label{classes:catdata.CATData.get_free_atom_energies}\pysiglinewithargsret{\bfcode{get\_free\_atom\_energies}}{}{}
Returns a ditionary of free atom energies corresponding to each of 
the elements.

\end{fulllineitems}

\index{get\_ground\_state() (catdata.CATData method)}

\begin{fulllineitems}
\phantomsection\label{classes:catdata.CATData.get_ground_state}\pysiglinewithargsret{\bfcode{get\_ground\_state}}{\emph{elements=None}, \emph{out='atoms'}, \emph{spin=0}}{}
Returns the lowest energy structure containing all the elements
in elements.

For alloys, energy of formation is calculated.
\begin{quote}\begin{description}
\item[{Parameters}] \leavevmode\begin{itemize}
\item {} \begin{description}
\item[{\emph{elements}: list}] \leavevmode
list of chemical symbols

\end{description}

\item {} \begin{description}
\item[{\emph{out}: str}] \leavevmode
directive for output, if `atoms' will return the ASE Atoms
object, otherwise will return info{[}out{]}

\end{description}

\item {} \begin{description}
\item[{\emph{spin}: int}] \leavevmode
\code{0} : all   \code{1} : only non-mag  \code{2} : only mag

\end{description}

\end{itemize}

\end{description}\end{quote}

\end{fulllineitems}

\index{get\_parent() (catdata.CATData method)}

\begin{fulllineitems}
\phantomsection\label{classes:catdata.CATData.get_parent}\pysiglinewithargsret{\bfcode{get\_parent}}{\emph{atom}}{}
Returns the system\_ID of the parent of atom.

If parent is not found will return system\_ID of atom.

\begin{notice}{note}{Note:}
Only considers bulk atoms with different volumes from parent 
but should be generalized for other calculation types
However, for the structure map, even if two atoms are related
but have different cells, e.g transformation path they will
have different moments hence should be treated as separate
structures.
\end{notice}
\begin{quote}\begin{description}
\item[{Parameters}] \leavevmode\begin{itemize}
\item {} 
\emph{atom}: instance of \titleref{ase.Atoms}
\begin{quote}

child of the parent atom
\end{quote}

\end{itemize}

\end{description}\end{quote}

\end{fulllineitems}

\index{get\_ref\_atoms() (catdata.CATData method)}

\begin{fulllineitems}
\phantomsection\label{classes:catdata.CATData.get_ref_atoms}\pysiglinewithargsret{\bfcode{get\_ref\_atoms}}{\emph{structures=None}, \emph{quantities=None}, \emph{sort\_by=None}, \emph{refene=None}}{}
Returns list of ASE atoms objects specified in structures and with
property in quantities.
\begin{quote}\begin{description}
\item[{Parameters}] \leavevmode\begin{itemize}
\item {} 
\emph{structures}: list
\begin{quote}

list of strings of strucname of part of strucname, e.g. `bcc' 
or system\_ID, wildcards are permitted, e.g. \code{Fe*/*/*/1/*}
(see definition of system\_ID)
\end{quote}

\item {} 
\emph{quantities}: list
\begin{quote}

list of desired properties that a structure must possess to be 
included. If a structure has more than one property, it will
be included multiple times.

if quantities is None will simply return what is stored in 
memory.
\end{quote}

\end{itemize}

\end{description}\end{quote}

\end{fulllineitems}

\index{get\_ref\_data() (catdata.CATData method)}

\begin{fulllineitems}
\phantomsection\label{classes:catdata.CATData.get_ref_data}\pysiglinewithargsret{\bfcode{get\_ref\_data}}{\emph{structures=None}, \emph{quantities=None}}{}
Returns the properties of the reference atoms. 
See {\hyperref[classes:catdata.CATData.get_ref_atoms]{\crossref{\code{get\_ref\_atoms()}}}} for description of parameters.

If quantities is None, it will return what is stored in memory,
otherwise, it will initialize ref\_atoms if quantities is not
the same as stored.

\end{fulllineitems}

\index{get\_ref\_weights() (catdata.CATData method)}

\begin{fulllineitems}
\phantomsection\label{classes:catdata.CATData.get_ref_weights}\pysiglinewithargsret{\bfcode{get\_ref\_weights}}{\emph{structures=None}, \emph{quantities=None}}{}
Returns the weights of the reference atoms in the optimization. 
See {\hyperref[classes:catdata.CATData.get_ref_atoms]{\crossref{\code{get\_ref\_atoms()}}}} for description of parameters.

If quantities is None, it will return what is stored in memory,
otherwise, it will initialize ref\_atoms if quantities is not
the same as stored.

If any of the reference atoms has no weight then all the weights
will be nullified.

\end{fulllineitems}

\index{get\_structuremap\_distance() (catdata.CATData method)}

\begin{fulllineitems}
\phantomsection\label{classes:catdata.CATData.get_structuremap_distance}\pysiglinewithargsret{\bfcode{get\_structuremap\_distance}}{\emph{struc0}, \emph{strucs}, \emph{index='total'}}{}
Returns the distance of the strucs from struc0 in the 
structure map
\begin{quote}\begin{description}
\item[{Parameters}] \leavevmode\begin{itemize}
\item {} 
\emph{struc0}: str
\begin{quote}

strucname,  e.g. \titleref{`dimer'}

system\_ID, e.g. \titleref{`Fe/229/0/1/*'} : all bcc-Fe E-V structures

ASE Atoms object, coordinates is expected to be normalized
globally
\end{quote}

\item {} 
\emph{strucs}: list
\begin{quote}

list of structures (format similar to struc0)
\end{quote}

\item {} 
\emph{mode}: str
\begin{quote}

average/total/maximum of the distances
\end{quote}

\item {} 
\emph{index}: list, int, str
\begin{quote}

list of atom indices corresponding to each struc in strucs
whose distance from struc0 is returned, if int will apply 
same index to all strucs, if total will average the coordinates
\end{quote}

\end{itemize}

\end{description}\end{quote}

\end{fulllineitems}

\index{get\_structures() (catdata.CATData method)}

\begin{fulllineitems}
\phantomsection\label{classes:catdata.CATData.get_structures}\pysiglinewithargsret{\bfcode{get\_structures}}{\emph{name=False}}{}
Returns the system\_ID or strucname (name=True) of all structures.
The definition of system\_ID varies depending on data\_type of structure.

\end{fulllineitems}


\end{fulllineitems}



\section{Model}
\label{classes:model}\label{classes:module-catparam}\index{catparam (module)}\index{CATParam (class in catparam)}

\begin{fulllineitems}
\phantomsection\label{classes:catparam.CATParam}\pysiglinewithargsret{\strong{class }\code{catparam.}\bfcode{CATParam}}{\emph{**kwargs}}{}
Defines the model for optimization.

It includes functionalities to read or generate models.

The models are stored as list of model objects, e.g. modelsbx.

\begin{notice}{note}{Todo}

generalize models format for other calculators
\end{notice}
\begin{quote}\begin{description}
\item[{Parameters}] \leavevmode\begin{itemize}
\item {} 
\emph{controls}: instance of CATControls
\begin{quote}

CATControls object to initialize parameters
\end{quote}

\item {} 
\emph{elements}: list
\begin{quote}

list of chemical symbols
\end{quote}

\item {} 
\emph{model}: str
\begin{quote}

model version
\end{quote}

\item {} 
\emph{model\_filename}: str
\begin{quote}

file containing model, can also be path to file
\end{quote}

\item {} 
\emph{valences}: dict
\begin{quote}

dictionary of element:valency
\end{quote}

\item {} 
\emph{valenceelectrons}: dict
\begin{quote}

dictionary of element:number of valence electrons
\begin{quote}

\code{None}: will read valenceelectrons from 
\code{variables.atomic\_properties()}
\end{quote}
\end{quote}

\item {} 
\emph{functions}: dict
\begin{quote}

dictionary of keyword:bondpair:function or keyword:function, 
function is one of the instances in functions module. The latter
implies that all bond pairs will have the same functional form
\end{quote}

\item {} 
\emph{cutoff}: dict
\begin{quote}

dictionary of cutoff keyword:bondpair:value or keyword:value.
The latter implies that all bond pairs will have the same value
\end{quote}

\item {} 
\emph{pathtobetas}: str
\begin{quote}

path to betas files

\code{None}: will use \code{variables.pathtobetas()}
\end{quote}

\item {} 
\emph{pathtoonsites}: str
\begin{quote}

path to onsites files

\code{None}: will use \code{variables.pathtoonsites()}
\end{quote}

\item {} 
\emph{betafitstruc}: str
\begin{quote}

structure from which the betas were derived (see format of betas 
files)

\code{None}: will default to `dimer'
\end{quote}

\item {} 
\emph{betatype}: str
\begin{quote}

method used in beta generation

\code{None}: will default to `loewdin'
\end{quote}

\item {} 
\emph{betabasis}: str
\begin{quote}

basis used in beta generation 
\code{None}: will default to `tz0'
\end{quote}

\item {} 
\emph{calculator}: str
\begin{quote}

name of the calculator
\end{quote}

\item {} 
\emph{calculator\_settings}: dict
\begin{quote}

dictionary of controls passed to the calculator
\end{quote}

\item {} 
\emph{orthogonal}: bool
\begin{quote}

orthogonal or non-orthogonal model
\end{quote}

\item {} 
\emph{data}: instance of CATData
\begin{quote}

CATData object to required to generate models
\end{quote}

\end{itemize}

\end{description}\end{quote}
\index{average\_modelsbx() (catparam.CATParam method)}

\begin{fulllineitems}
\phantomsection\label{classes:catparam.CATParam.average_modelsbx}\pysiglinewithargsret{\bfcode{average\_modelsbx}}{\emph{iterations='all'}}{}~\begin{quote}

Returns a model by averaging models in iterations.
\end{quote}
\begin{quote}\begin{description}
\item[{Parameters}] \leavevmode\begin{itemize}
\item {} 
\emph{iterations} : list
\begin{quote}

indices of the models to be averaged

\titleref{all} : include all models
\end{quote}

\end{itemize}

\end{description}\end{quote}

\end{fulllineitems}

\index{calc\_rcut() (catparam.CATParam method)}

\begin{fulllineitems}
\phantomsection\label{classes:catparam.CATParam.calc_rcut}\pysiglinewithargsret{\bfcode{calc\_rcut}}{\emph{pair}, \emph{debug=False}}{}
Returns a dictionary of cut-off keyword: value.

\titleref{rcut} is set to the distance when the value of the most long-ranged 
bond integral is 5 percent its value at the equilibrium 
dimer distance. The hierarcy of the range of the bond integrals (m=0)
for simplicity is assumed as follows:

\code{ppsigma, spsigma, sssigma, pdsigma, sdsigma, ddsigma}

The cut-off function starts at 15 percent. \titleref{r2cut} is 
\titleref{1.5*rcut} and \titleref{d2cut} is \titleref{2*dcut}

It also sets the cutoff versions to cosine.
\begin{quote}\begin{description}
\item[{Parameters}] \leavevmode\begin{itemize}
\item {} 
\emph{bondpair}: list
\begin{quote}

pair of chemical symbols
\end{quote}

\item {} 
\emph{debug}: bool
\begin{quote}

if debug will plot the cut-off values with the reference 
bond integral
\end{quote}

\end{itemize}

\end{description}\end{quote}

\end{fulllineitems}

\index{gen\_jii() (catparam.CATParam method)}

\begin{fulllineitems}
\phantomsection\label{classes:catparam.CATParam.gen_jii}\pysiglinewithargsret{\bfcode{gen\_jii}}{\emph{elem}}{}
Returns the Jii parameter. This is read from 
\code{variables.atomic\_properties()}.
\begin{quote}\begin{description}
\item[{Parameters}] \leavevmode\begin{itemize}
\item {} 
\emph{elem}: str
\begin{quote}

chemical symbol
\end{quote}

\end{itemize}

\end{description}\end{quote}

\end{fulllineitems}

\index{gen\_model() (catparam.CATParam method)}

\begin{fulllineitems}
\phantomsection\label{classes:catparam.CATParam.gen_model}\pysiglinewithargsret{\bfcode{gen\_model}}{\emph{**kwargs}}{}~\begin{quote}

Returns model specific to the calculator.

Generates model. Calls calculator-specific function to generate model.

\begin{notice}{note}{Todo}

Extend to other calculators.
\end{notice}
\end{quote}
\begin{quote}\begin{description}
\item[{Parameters}] \leavevmode\begin{itemize}
\item {} 
\emph{kwargs} : dict
\begin{quote}

directives for future extensions.
\end{quote}

\end{itemize}

\end{description}\end{quote}

\end{fulllineitems}

\index{gen\_modelsbx() (catparam.CATParam method)}

\begin{fulllineitems}
\phantomsection\label{classes:catparam.CATParam.gen_modelsbx}\pysiglinewithargsret{\bfcode{gen\_modelsbx}}{\emph{**kwargs}}{}
Returns a modelsbx object. See {\hyperref[classes:bopmodel.modelsbx]{\crossref{\code{bopmodel.modelsbx}}}}

Generates modelsbx. Calls \code{\_gen\_atomsbx()} and \code{\_gen\_bondsbx()}
\begin{quote}\begin{description}
\item[{Parameters}] \leavevmode\begin{itemize}
\item {} 
\emph{kwargs} : dict
\begin{quote}

directives for future extensions. For example, \titleref{part} can be 
set to \titleref{bond}, \titleref{rep} or \titleref{all} to specify part of modelsbx that 
is generated.
\end{quote}

\end{itemize}

\end{description}\end{quote}

\end{fulllineitems}

\index{gen\_onsite() (catparam.CATParam method)}

\begin{fulllineitems}
\phantomsection\label{classes:catparam.CATParam.gen_onsite}\pysiglinewithargsret{\bfcode{gen\_onsite}}{\emph{elem}}{}
Returns a dictionary of the valence:onsites read from the onsite file.

It calls the function \code{\_read\_onsites()}.

The filename must follow the format
\begin{quote}

\code{{[}structure{]}\_{[}elem{]}{[}elem{]}\_{[}basis{]}\_{[}betatype{]}.onsites}
\end{quote}

The structure, basis and betatype are similar to the bond integrals.
The onsite is taken as the average over a range of distances.

\begin{notice}{note}{Todo}

fit distance dependence of onsites
\end{notice}

\begin{notice}{note}{Note:}
Determine if onsite should just be set to zero.
\end{notice}
\begin{quote}\begin{description}
\item[{Parameters}] \leavevmode\begin{itemize}
\item {} 
\emph{elem}: str
\begin{quote}

chemical symbol
\end{quote}

\end{itemize}

\end{description}\end{quote}

\end{fulllineitems}

\index{gen\_stoner() (catparam.CATParam method)}

\begin{fulllineitems}
\phantomsection\label{classes:catparam.CATParam.gen_stoner}\pysiglinewithargsret{\bfcode{gen\_stoner}}{\emph{elem}, \emph{struc='gs'}}{}
Returns the Stoner parameter. This is read from 
\code{variables.atomic\_properties()}.

..todo:: fit to mag-non mag energy difference of struc
\begin{quote}\begin{description}
\item[{Parameters}] \leavevmode\begin{itemize}
\item {} 
\emph{elem}: str
\begin{quote}

chemical symbol
\end{quote}

\end{itemize}

\end{description}\end{quote}

\end{fulllineitems}

\index{gen\_valenceelectrons() (catparam.CATParam method)}

\begin{fulllineitems}
\phantomsection\label{classes:catparam.CATParam.gen_valenceelectrons}\pysiglinewithargsret{\bfcode{gen\_valenceelectrons}}{\emph{elem}}{}
Returns the number of valence electrons and orbitals. This is read from
\code{variables.atomic\_properties()}.
\begin{quote}\begin{description}
\item[{Parameters}] \leavevmode\begin{itemize}
\item {} 
\emph{elem}: str
\begin{quote}

chemical symbol
\end{quote}

\end{itemize}

\end{description}\end{quote}

\end{fulllineitems}

\index{get\_SED() (catparam.CATParam method)}

\begin{fulllineitems}
\phantomsection\label{classes:catparam.CATParam.get_SED}\pysiglinewithargsret{\bfcode{get\_SED}}{\emph{elem}, \emph{model}, \emph{strucs=None}}{}
Returns an array of the bond energy differences of the structures
in strucs. The reference is the first structure not necessarily the
lowest in energy in order to simplify fitting where one does not
need the index of the lowest energy structure.

Calls \code{sedt.energy\_diff()} to calculate bond energies of 
structures adjusted to have the same average second moment.
\begin{quote}\begin{description}
\item[{Parameters}] \leavevmode\begin{itemize}
\item {} 
\emph{elem} : str
\begin{quote}

chemical symbol
\end{quote}

\item {} 
\emph{model} : calculator-specific object
\begin{quote}

model to be used by the calculator
\end{quote}

\item {} 
\emph{strucs} : list
\begin{quote}

list of system\_ID's corresponding to relaxed structures

\titleref{None}: will include all relaxed structures
\end{quote}

\end{itemize}

\end{description}\end{quote}

\end{fulllineitems}

\index{get\_atoms\_properties() (catparam.CATParam method)}

\begin{fulllineitems}
\phantomsection\label{classes:catparam.CATParam.get_atoms_properties}\pysiglinewithargsret{\bfcode{get\_atoms\_properties}}{\emph{elem}}{}
Returns the mass, number of orbitals and valence electrons, Jii,
onsite and Stoner parameters. This calls {\hyperref[classes:catparam.CATParam.gen_valenceelectrons]{\crossref{\code{gen\_valenceelectrons()}}}},
{\hyperref[classes:catparam.CATParam.gen_jii]{\crossref{\code{gen\_jii()}}}}, {\hyperref[classes:catparam.CATParam.gen_stoner]{\crossref{\code{gen\_stoner()}}}}, {\hyperref[classes:catparam.CATParam.gen_onsite]{\crossref{\code{gen\_onsite()}}}}.
\begin{quote}\begin{description}
\item[{Parameters}] \leavevmode\begin{itemize}
\item {} 
\emph{elem}: str
\begin{quote}

chemical symbol
\end{quote}

\end{itemize}

\end{description}\end{quote}

\end{fulllineitems}

\index{get\_cutoffpara() (catparam.CATParam method)}

\begin{fulllineitems}
\phantomsection\label{classes:catparam.CATParam.get_cutoffpara}\pysiglinewithargsret{\bfcode{get\_cutoffpara}}{\emph{bondpair}}{}
Returns a dictionary of cut-off keyword: value. Calls {\hyperref[classes:catparam.CATParam.calc_rcut]{\crossref{\code{calc\_rcut()}}}}
\begin{quote}\begin{description}
\item[{Parameters}] \leavevmode\begin{itemize}
\item {} 
\emph{bondpair}: list
\begin{quote}

pair of chemical symbols
\end{quote}

\end{itemize}

\end{description}\end{quote}

\end{fulllineitems}

\index{get\_fitfunc() (catparam.CATParam method)}

\begin{fulllineitems}
\phantomsection\label{classes:catparam.CATParam.get_fitfunc}\pysiglinewithargsret{\bfcode{get\_fitfunc}}{\emph{bondpair}}{}
Returns a dictionary of keyword:functions
\begin{quote}\begin{description}
\item[{Parameters}] \leavevmode\begin{itemize}
\item {} 
\emph{bondpair}: list
\begin{quote}

pair of chemical symbols
\end{quote}

\end{itemize}

\end{description}\end{quote}

\end{fulllineitems}

\index{get\_initial\_bondparam() (catparam.CATParam method)}

\begin{fulllineitems}
\phantomsection\label{classes:catparam.CATParam.get_initial_bondparam}\pysiglinewithargsret{\bfcode{get\_initial\_bondparam}}{\emph{bondpair}}{}
Returns a dictionary of keyword:functions resulting from the initial
fitting of the bond/overlap integrals.
\begin{quote}\begin{description}
\item[{Parameters}] \leavevmode\begin{itemize}
\item {} 
\emph{bondpair}: list
\begin{quote}

pair of chemical symbols
\end{quote}

\end{itemize}

\end{description}\end{quote}

\end{fulllineitems}

\index{get\_initial\_repparam() (catparam.CATParam method)}

\begin{fulllineitems}
\phantomsection\label{classes:catparam.CATParam.get_initial_repparam}\pysiglinewithargsret{\bfcode{get\_initial\_repparam}}{\emph{bondpair}}{}
Returns a dictionary of keyword:functions resulting from the initial
fitting of the dimer repulsive energy
\begin{quote}\begin{description}
\item[{Parameters}] \leavevmode\begin{itemize}
\item {} 
\emph{bondpair}: list
\begin{quote}

pair of chemical symbols
\end{quote}

\end{itemize}

\end{description}\end{quote}

\end{fulllineitems}

\index{get\_model() (catparam.CATParam method)}

\begin{fulllineitems}
\phantomsection\label{classes:catparam.CATParam.get_model}\pysiglinewithargsret{\bfcode{get\_model}}{\emph{iteration=0}}{}~\begin{quote}

Returns the model generated at N=iteration step.
\end{quote}
\begin{quote}\begin{description}
\item[{Parameters}] \leavevmode\begin{itemize}
\item {} 
\emph{iteration} : int
\begin{quote}

index of the model in list
\end{quote}

\end{itemize}

\end{description}\end{quote}

\end{fulllineitems}

\index{read\_model() (catparam.CATParam method)}

\begin{fulllineitems}
\phantomsection\label{classes:catparam.CATParam.read_model}\pysiglinewithargsret{\bfcode{read\_model}}{\emph{**kwargs}}{}~\begin{quote}

Returns model specific to the calculator.

Reads model. Calls calculator-specific function to read model.

\begin{notice}{note}{Todo}

Extend to other calculators.
\end{notice}
\end{quote}
\begin{quote}\begin{description}
\item[{Parameters}] \leavevmode\begin{itemize}
\item {} 
\emph{kwargs} : dict
\begin{quote}

directives for future extensions.
\end{quote}

\end{itemize}

\end{description}\end{quote}

\end{fulllineitems}

\index{read\_modelsbx() (catparam.CATParam method)}

\begin{fulllineitems}
\phantomsection\label{classes:catparam.CATParam.read_modelsbx}\pysiglinewithargsret{\bfcode{read\_modelsbx}}{\emph{**kwargs}}{}
Returns a modelsbx object. See {\hyperref[classes:bopmodel.modelsbx]{\crossref{\code{bopmodel.modelsbx}}}}

Reads modelsbx. Calls \code{bopmodel.read\_modelsbx()}
\begin{quote}\begin{description}
\item[{Parameters}] \leavevmode\begin{itemize}
\item {} 
\emph{kwargs} : dict
\begin{quote}

directives for future extensions.
\end{quote}

\end{itemize}

\end{description}\end{quote}

\end{fulllineitems}


\end{fulllineitems}

\phantomsection\label{classes:module-beta}\index{beta (module)}\index{Beta (class in beta)}

\begin{fulllineitems}
\phantomsection\label{classes:beta.Beta}\pysiglinewithargsret{\strong{class }\code{beta.}\bfcode{Beta}}{\emph{**kwargs}}{}
Defines a set of bond integrals for constructing the hamiltonian.

It reads the data points from a file stored in pathtobetas. 
A set of functions in fitfuncs is parametrized with respect to the data.

The betas file should be named as 
{[}struc{]}\_{[}ele1{]}{[}ele2{]}\_{[}basis{]}\_{[}betatype{]}.betas

See the folder betas in the examples folder for reference.
\begin{quote}\begin{description}
\item[{Parameters}] \leavevmode\begin{itemize}
\item {} 
\emph{structure}: str
\begin{quote}

structure from which the bond integrals were derived
\end{quote}

\item {} 
\emph{bondpair}: tuple
\begin{quote}

pair of chemical symbols
\end{quote}

\item {} 
\emph{basis}: str
\begin{quote}

basis used for downfolding
\end{quote}

\item {} 
\emph{betatype}: str
\begin{quote}

can be unscreened, overlap, loewdin or other beta type
\end{quote}

\item {} 
\emph{valence}: tuple
\begin{quote}

pair of valency, e.g. \code{sd}
\end{quote}

\item {} 
\emph{pathtobetas}: str
\begin{quote}

path to betas file
\end{quote}

\end{itemize}

\end{description}\end{quote}
\index{read\_betas() (beta.Beta method)}

\begin{fulllineitems}
\phantomsection\label{classes:beta.Beta.read_betas}\pysiglinewithargsret{\bfcode{read\_betas}}{}{}
Reads in bond integrals from a betamaker output file(.betas) 
saved in pathtobeta. If not provided, will read from folder:
/betas. The filenames must follow the format: 
{[}structure{]}\_{[}elementA{]}{[}elementB{]}\_{[}basis{]}\_{[}betatype{]}.betas

\end{fulllineitems}

\index{set\_fitfuncs() (beta.Beta method)}

\begin{fulllineitems}
\phantomsection\label{classes:beta.Beta.set_fitfuncs}\pysiglinewithargsret{\bfcode{set\_fitfuncs}}{\emph{fitfuncs}}{}
Set fitting function , default(sum\_exponential)

\end{fulllineitems}

\index{zip\_betas() (beta.Beta method)}

\begin{fulllineitems}
\phantomsection\label{classes:beta.Beta.zip_betas}\pysiglinewithargsret{\bfcode{zip\_betas}}{}{}
Returs a dictionary of the bond integrals

\end{fulllineitems}


\end{fulllineitems}



\section{Calculator}
\label{classes:calculator}\label{classes:module-catcalc}\index{catcalc (module)}\index{CATCalc (class in catcalc)}

\begin{fulllineitems}
\phantomsection\label{classes:catcalc.CATCalc}\pysiglinewithargsret{\strong{class }\code{catcalc.}\bfcode{CATCalc}}{\emph{**kwargs}}{}
Defines a set of calculators to determine required properties.

The calculators are stored as list of instances of the calculator.
\begin{quote}\begin{description}
\item[{Parameters}] \leavevmode\begin{itemize}
\item {} 
\emph{calculator}: str
\begin{quote}

name of the calculator
\end{quote}

\item {} 
\emph{calculator\_settings}: dict
\begin{quote}

dictionary of calculator-specific parameters
\end{quote}

\item {} 
\emph{nproc}: int
\begin{quote}

number of parallel processes

\code{None}: will not parallelize

\code{default}: number of cpu * 2
\end{quote}

\item {} 
\emph{controls}: instance of CATControls
\begin{quote}

CATControls object to initialize parameters
\end{quote}

\item {} 
\emph{parallel}: string
\begin{quote}

serial, multiprocessing, mpi
\end{quote}

\end{itemize}

\end{description}\end{quote}
\index{calc\_def\_ene() (catcalc.CATCalc static method)}

\begin{fulllineitems}
\phantomsection\label{classes:catcalc.CATCalc.calc_def_ene}\pysiglinewithargsret{\strong{static }\bfcode{calc\_def\_ene}}{\emph{atom}, \emph{kwargs}}{}
Returns the Atoms object with the calculated defect energy.

The reference bulk structures are in atom.info{[}'reference\_atoms'{]}.
\begin{quote}\begin{description}
\item[{Parameters}] \leavevmode\begin{itemize}
\item {} 
\emph{atom}: instance of ASE Atoms object
\begin{quote}

structure to calculate
\end{quote}

\item {} 
\emph{kwargs}: dict
\begin{quote}

directives for calculator
\end{quote}

\end{itemize}

\end{description}\end{quote}

\end{fulllineitems}

\index{calc\_ebs() (catcalc.CATCalc static method)}

\begin{fulllineitems}
\phantomsection\label{classes:catcalc.CATCalc.calc_ebs}\pysiglinewithargsret{\strong{static }\bfcode{calc\_ebs}}{\emph{atom}, \emph{kwargs}}{}
Returns the Atoms object with the calculated eigenvalues.
\begin{quote}\begin{description}
\item[{Parameters}] \leavevmode\begin{itemize}
\item {} 
\emph{atom}: instance of ASE Atoms object
\begin{quote}

structure to calculate
\end{quote}

\item {} 
\emph{kwargs}: dict
\begin{quote}

directives for calculator
\end{quote}

\end{itemize}

\end{description}\end{quote}

\end{fulllineitems}

\index{calc\_efs() (catcalc.CATCalc static method)}

\begin{fulllineitems}
\phantomsection\label{classes:catcalc.CATCalc.calc_efs}\pysiglinewithargsret{\strong{static }\bfcode{calc\_efs}}{\emph{atom}, \emph{kwargs}}{}
Returns the Atoms object with the calculated energy, forces and 
stresses.
\begin{quote}\begin{description}
\item[{Parameters}] \leavevmode\begin{itemize}
\item {} 
\emph{atom}: instance of ASE Atoms object
\begin{quote}

structure to calculate
\end{quote}

\item {} 
\emph{kwargs}: dict
\begin{quote}

directives for calculator
\end{quote}

\end{itemize}

\end{description}\end{quote}

\end{fulllineitems}

\index{calculate() (catcalc.CATCalc static method)}

\begin{fulllineitems}
\phantomsection\label{classes:catcalc.CATCalc.calculate}\pysiglinewithargsret{\strong{static }\bfcode{calculate}}{\emph{atom}, \emph{kwargs}}{}
Returns the Atoms object with the calculated property.

The required property is defined by atom.info{[}'required\_property'{]}

Calls {\hyperref[classes:catcalc.CATCalc.calc_ebs]{\crossref{\code{calc\_ebs()}}}}, {\hyperref[classes:catcalc.CATCalc.calc_efs]{\crossref{\code{calc\_efs()}}}}, {\hyperref[classes:catcalc.CATCalc.calc_def_ene]{\crossref{\code{calc\_def\_ene()}}}}
\begin{quote}\begin{description}
\item[{Parameters}] \leavevmode\begin{itemize}
\item {} 
\emph{atom}: instance of ASE Atoms object
\begin{quote}

structure to calculate
\end{quote}

\item {} 
\emph{kwargs}: dict
\begin{quote}

directives for calculator
\end{quote}

\end{itemize}

\end{description}\end{quote}

\end{fulllineitems}

\index{get\_atoms() (catcalc.CATCalc method)}

\begin{fulllineitems}
\phantomsection\label{classes:catcalc.CATCalc.get_atoms}\pysiglinewithargsret{\bfcode{get\_atoms}}{}{}
Returns a list of the structures each an ASE Atoms object
assigned for calculation.

\end{fulllineitems}

\index{get\_calculator() (catcalc.CATCalc method)}

\begin{fulllineitems}
\phantomsection\label{classes:catcalc.CATCalc.get_calculator}\pysiglinewithargsret{\bfcode{get\_calculator}}{\emph{i}, \emph{update}}{}
Returns the calculator. If None will initialize 
the calculator.
\begin{quote}\begin{description}
\item[{Parameters}] \leavevmode\begin{itemize}
\item {} 
\emph{update}: bool
\begin{quote}

True: will update the model used by the calculator
\end{quote}

\end{itemize}

\end{description}\end{quote}

\end{fulllineitems}

\index{get\_calculators() (catcalc.CATCalc method)}

\begin{fulllineitems}
\phantomsection\label{classes:catcalc.CATCalc.get_calculators}\pysiglinewithargsret{\bfcode{get\_calculators}}{}{}
Returns a list of calculators corresponding to each structure

calls {\hyperref[classes:catcalc.CATCalc.get_calculator]{\crossref{\code{get\_calculator()}}}}

\end{fulllineitems}

\index{get\_property() (catcalc.CATCalc method)}

\begin{fulllineitems}
\phantomsection\label{classes:catcalc.CATCalc.get_property}\pysiglinewithargsret{\bfcode{get\_property}}{\emph{**kwargs}}{}
Returns list of calculated properties for all structures.

Calls {\hyperref[classes:catcalc.CATCalc.calculate]{\crossref{\code{calculate()}}}}
\begin{quote}\begin{description}
\item[{Parameters}] \leavevmode\begin{itemize}
\item {} 
\emph{kwargs}: dict
\begin{quote}

directives for calculator
\end{quote}

\end{itemize}

\end{description}\end{quote}

\end{fulllineitems}

\index{get\_structures() (catcalc.CATCalc method)}

\begin{fulllineitems}
\phantomsection\label{classes:catcalc.CATCalc.get_structures}\pysiglinewithargsret{\bfcode{get\_structures}}{}{}
Returns a list of the system ID of all the structures

\end{fulllineitems}

\index{pack\_atoms() (catcalc.CATCalc method)}

\begin{fulllineitems}
\phantomsection\label{classes:catcalc.CATCalc.pack_atoms}\pysiglinewithargsret{\bfcode{pack\_atoms}}{}{}
Pack all properties for the same structure so do only one calculation
of all properties for one structure.

\end{fulllineitems}

\index{set\_atoms() (catcalc.CATCalc method)}

\begin{fulllineitems}
\phantomsection\label{classes:catcalc.CATCalc.set_atoms}\pysiglinewithargsret{\bfcode{set\_atoms}}{\emph{atoms}}{}
Set the structures for calculation. The calculators
and results are reset.
\begin{quote}\begin{description}
\item[{Parameters}] \leavevmode\begin{itemize}
\item {} 
\emph{atoms}: list
\begin{quote}

list of ASE Atoms objects
\end{quote}

\end{itemize}

\end{description}\end{quote}

\end{fulllineitems}

\index{set\_model() (catcalc.CATCalc method)}

\begin{fulllineitems}
\phantomsection\label{classes:catcalc.CATCalc.set_model}\pysiglinewithargsret{\bfcode{set\_model}}{\emph{model}}{}
Set the model for the calcuator. The calculators
and results are reset.
\begin{quote}\begin{description}
\item[{Parameters}] \leavevmode\begin{itemize}
\item {} 
\emph{model}: instance of calculator-specific model
\begin{quote}

model used for the calculator
\end{quote}

\end{itemize}

\end{description}\end{quote}

\end{fulllineitems}


\end{fulllineitems}



\section{Optimization kernel}
\label{classes:optimization-kernel}\label{classes:module-catkernel}\index{catkernel (module)}\index{CATKernel (class in catkernel)}

\begin{fulllineitems}
\phantomsection\label{classes:catkernel.CATKernel}\pysiglinewithargsret{\strong{class }\code{catkernel.}\bfcode{CATKernel}}{\emph{**kwargs}}{}
Defines the optimization kernel.

It generates the objective function and the optimizer.
\begin{quote}\begin{description}
\item[{Parameters}] \leavevmode\begin{itemize}
\item {} 
\emph{controls}: instance of CATControls
\begin{quote}

CATControls object to initialize parameters
\end{quote}

\item {} 
\emph{calc}: instance of CATcalc
\begin{quote}

CATcalc object to define the calculator
\end{quote}

\item {} 
\emph{ref\_data}: list
\begin{quote}

list of calculated properties
\end{quote}

\item {} 
\emph{variables}: list
\begin{quote}

list of dictionary of contraints on the parameters
\end{quote}

\item {} 
\emph{objective}: callable object
\begin{quote}

should take parameters as argument and return the error

None: will generate default objective function

str: name of the objective function, should be implemented in 
{\hyperref[classes:catkernel.CATobjective]{\crossref{\code{CATobjective}}}}
\end{quote}

\item {} 
\emph{optimizer}: callable object
\begin{quote}

should return optimized parameters

str: name of the optimizer, should be implemented in 
{\hyperref[classes:catkernel.CAToptimizer]{\crossref{\code{CAToptimizer}}}}
\end{quote}

\item {} 
\emph{optimizer\_options}: dict
\begin{quote}

controls to be passed to the optimizer
\end{quote}

\item {} 
\emph{weights}: list
\begin{quote}

weight of each structure in the error

None: will assign weight from \code{variables.fit\_weights()}
\end{quote}

\item {} 
\emph{log}: str
\begin{quote}

output file for writing parameters and error at each optimization
step
\end{quote}

\item {} 
\emph{verbose}: int
\begin{quote}

controls verbosity e.g.

prints on screen rms at each optimization step if verbose \textgreater{}= 1
\end{quote}

\item {} 
\emph{dump\_min\_model}: bool
\begin{quote}

logical flag to switch on dumping of best model at each iteration
\end{quote}

\end{itemize}

\end{description}\end{quote}
\index{gen\_objective() (catkernel.CATKernel method)}

\begin{fulllineitems}
\phantomsection\label{classes:catkernel.CATKernel.gen_objective}\pysiglinewithargsret{\bfcode{gen\_objective}}{}{}
''
Generates the objective function. Standard setting is unknown.

\end{fulllineitems}

\index{get\_optimized\_model() (catkernel.CATKernel method)}

\begin{fulllineitems}
\phantomsection\label{classes:catkernel.CATKernel.get_optimized_model}\pysiglinewithargsret{\bfcode{get\_optimized\_model}}{}{}
Returns the optimized model.

\end{fulllineitems}

\index{optimize() (catkernel.CATKernel method)}

\begin{fulllineitems}
\phantomsection\label{classes:catkernel.CATKernel.optimize}\pysiglinewithargsret{\bfcode{optimize}}{}{}
Builds objective and optimizer objects if not initialized and 
runs optimization. The resulting model is assigned to optimized\_model.

\end{fulllineitems}

\index{test() (catkernel.CATKernel method)}

\begin{fulllineitems}
\phantomsection\label{classes:catkernel.CATKernel.test}\pysiglinewithargsret{\bfcode{test}}{\emph{param\_weights=None}}{}
Calculates the error for current model with parameters weighted by
param\_weights

\end{fulllineitems}

\index{write\_summary() (catkernel.CATKernel method)}

\begin{fulllineitems}
\phantomsection\label{classes:catkernel.CATKernel.write_summary}\pysiglinewithargsret{\bfcode{write\_summary}}{}{}
Prints to screen the starting and ending parameters.

\end{fulllineitems}


\end{fulllineitems}

\index{CATobjective (class in catkernel)}

\begin{fulllineitems}
\phantomsection\label{classes:catkernel.CATobjective}\pysiglinewithargsret{\strong{class }\code{catkernel.}\bfcode{CATobjective}}{\emph{**kwargs}}{}
Defines the objective function. Callable with the parameters as argument, 
will return the corresponding error.
\begin{quote}\begin{description}
\item[{Parameters}] \leavevmode\begin{itemize}
\item {} 
\emph{calc}: instance of CATcalc
\begin{quote}

CATcalc object to define the calculator
\end{quote}

\item {} 
\emph{ref\_data}: list
\begin{quote}

list of calculated properties
\end{quote}

\item {} 
\emph{variables}: list
\begin{quote}

list of dictionary of contraints on the parameters
\end{quote}

\item {} 
\emph{penalty\_coeffs}: list
\begin{quote}

list of penalty coefficients for individual coefficients
\end{quote}

\item {} 
\emph{weights}: list
\begin{quote}

weight of each structure in the error

None: will assign weight from \code{variables.fit\_weights()}
\end{quote}

\item {} 
\emph{log}: str
\begin{quote}

output file for writing parameters and error at each optimization
step
\end{quote}

\item {} 
\emph{error\_vec\_log}: str
\begin{quote}

output file for writing error for each structure at each 
optimization step
\end{quote}

\item {} 
\emph{verbose}: bool
\begin{quote}

prints on screen rms at each optimization step
\end{quote}

\item {} 
\emph{array}: bool
\begin{quote}

if error is vector or scalar
\end{quote}

\item {} 
\emph{name}: str
\begin{quote}

name of the objective function
\end{quote}

\item {} 
\emph{dump\_min\_model}: bool
\begin{quote}

logical flag to switch on dumping of best model at each iteration
\end{quote}

\end{itemize}

\end{description}\end{quote}
\index{default() (catkernel.CATobjective method)}

\begin{fulllineitems}
\phantomsection\label{classes:catkernel.CATobjective.default}\pysiglinewithargsret{\bfcode{default}}{\emph{x0}}{}
Default objective function.

Returns the diffferences between the calculated properties and the
reference.

\end{fulllineitems}

\index{difference() (catkernel.CATobjective method)}

\begin{fulllineitems}
\phantomsection\label{classes:catkernel.CATobjective.difference}\pysiglinewithargsret{\bfcode{difference}}{\emph{x0}}{}
Objective function for energy differences calculations.

Returns the differences between delta(calculated properties) and
delta(reference):

\begin{Verbatim}[commandchars=\\\{\}]
\PYG{n}{delta\PYGZus{}data} \PYG{o}{=} \PYG{n}{data}\PYG{p}{[}\PYG{p}{:}\PYG{n+nb}{len}\PYG{p}{(}\PYG{n}{data}\PYG{p}{)}\PYG{o}{/}\PYG{l+m+mi}{2}\PYG{p}{]} \PYG{o}{\PYGZhy{}} \PYG{n}{data}\PYG{p}{[}\PYG{n+nb}{len}\PYG{p}{(}\PYG{n}{data}\PYG{p}{)}\PYG{o}{/}\PYG{l+m+mi}{2}\PYG{p}{:}\PYG{p}{]}
\end{Verbatim}

\end{fulllineitems}

\index{is\_bond\_required() (catkernel.CATobjective method)}

\begin{fulllineitems}
\phantomsection\label{classes:catkernel.CATobjective.is_bond_required}\pysiglinewithargsret{\bfcode{is\_bond\_required}}{}{}
Checks if bond-related parameters are included in fit

\end{fulllineitems}

\index{write() (catkernel.CATobjective method)}

\begin{fulllineitems}
\phantomsection\label{classes:catkernel.CATobjective.write}\pysiglinewithargsret{\bfcode{write}}{}{}
Dumps the parameters and errors at each step on files.

\end{fulllineitems}


\end{fulllineitems}

\index{CAToptimizer (class in catkernel)}

\begin{fulllineitems}
\phantomsection\label{classes:catkernel.CAToptimizer}\pysiglinewithargsret{\strong{class }\code{catkernel.}\bfcode{CAToptimizer}}{\emph{**kwargs}}{}
Defines the optimizer. Callable and returns the optimized parameters.
\begin{quote}\begin{description}
\item[{Parameters}] \leavevmode\begin{itemize}
\item {} 
\emph{objective}: callable object
\begin{quote}

should take parameters as argument and return the error
\end{quote}

\item {} 
\emph{x0}: list
\begin{quote}

initial guess for parameters
\end{quote}

\item {} 
\emph{optimizer\_options}: dict
\begin{quote}

controls to be passed to the optimizer
\end{quote}

\item {} 
\emph{name}: str
\begin{quote}

name of the optimizer
\end{quote}

\end{itemize}

\end{description}\end{quote}

\end{fulllineitems}



\section{BOPfox-ASE interface}
\label{classes:module-bopcal}\label{classes:bopfox-ase-interface}\index{bopcal (module)}\index{BOPfox (class in bopcal)}

\begin{fulllineitems}
\phantomsection\label{classes:bopcal.BOPfox}\pysiglinewithargsret{\strong{class }\code{bopcal.}\bfcode{BOPfox}}{\emph{restart=None}, \emph{track\_output=False}, \emph{atoms=None}, \emph{atomsbx=None}, \emph{bondsbx=None}, \emph{infoxbx='infox.bx'}, \emph{modelsbx='models.bx'}, \emph{bopfox='bopfox'}, \emph{savelog=False}, \emph{mem\_limit=None}, \emph{ignore\_errors=True}, \emph{debug=False}, \emph{kpoints=None}, \emph{root\_tmp\_folder='/tmp'}, \emph{magconfig=None}, \emph{**kwargs}}{}
Defines an ASE interface to BOPfox.

In order to define a calculator, the user should provide a modelsbx,
or atomsbx and bondsbx. These can either be objects (see \code{modelsbx},
\code{atomsbx} and \code{bondsbx}), filenames or paths. In addition, bopfox
input controls (infox parameters) can be optionally provided, otherwise the
calculator is expecting an infox.bx file on current working directory. 
The calculator has other attributes on top
of those of ASE such as {\hyperref[classes:bopcal.BOPfox.get_moments]{\crossref{\code{get\_moments()}}}}

\begin{notice}{note}{Todo}

extend ASE Atoms object to call BOPfox native functions.
\end{notice}
\begin{quote}\begin{description}
\item[{Parameters}] \leavevmode\begin{itemize}
\item {} 
\emph{atomsbx}: str or instance of atomsbx
\begin{quote}

used to create atoms.bx

\code{None}: will use modelsbx
\end{quote}

\item {} 
\emph{bondssbx}: str or instance of bondsbx
\begin{quote}

used to create bonds.bx

\code{None}: will use modelsbx
\end{quote}

\item {} 
\emph{modelsbx}: str or instance of modelsbx
\begin{quote}

used to create models.bx

\code{None}: will use modelsbx
\end{quote}

\item {} 
\emph{bopfox}: str
\begin{quote}

bopfox executable
\end{quote}

\item {} 
\emph{savelog}: bool
\begin{quote}

logical flag to save log file for further anaylis
\end{quote}

\item {} 
\emph{mem\_limit}: float
\begin{quote}

sets memory limit for bopfox calculation
\end{quote}

\item {} 
\emph{ignore\_errors}: bool
\begin{quote}

logical flag to skip bopfox errors
\end{quote}

\item {} 
\emph{root\_tmp\_folder}: str
\begin{quote}

directory where a tmp folder will be generated
\end{quote}

\item {} 
\emph{debug}: bool
\begin{quote}

logical flag for debugging purposes, will not delete temporary
folders
\end{quote}

\item {} 
\emph{kpoints}: list
\begin{quote}

coordinates of kpoints to be used for tight-binding band structure
calculation
\end{quote}

\item {} 
\emph{kwargs}:
\begin{quote}

optional keyword arguments to be written in infox, if not provided
will use infox.bx
\end{quote}

\end{itemize}

\end{description}\end{quote}
\index{calculate() (bopcal.BOPfox method)}

\begin{fulllineitems}
\phantomsection\label{classes:bopcal.BOPfox.calculate}\pysiglinewithargsret{\bfcode{calculate}}{\emph{*args}}{}
Sets up a temporary foler for perfoming a BOPfox calculation. Then 
writes the corresponding files and performs a calculation. 
Cleans up afterwards.

\end{fulllineitems}

\index{get\_mode() (bopcal.BOPfox method)}

\begin{fulllineitems}
\phantomsection\label{classes:bopcal.BOPfox.get_mode}\pysiglinewithargsret{\bfcode{get\_mode}}{}{}
Returns the  calculation mode (string).

\end{fulllineitems}

\index{get\_moments() (bopcal.BOPfox method)}

\begin{fulllineitems}
\phantomsection\label{classes:bopcal.BOPfox.get_moments}\pysiglinewithargsret{\bfcode{get\_moments}}{\emph{atom\_index=0}, \emph{moment=2}}{}
Returns the moments.

If the atom index is 0, then average will be returned
If the atom index is NOT an integer, it will return all moments

\end{fulllineitems}

\index{get\_name() (bopcal.BOPfox method)}

\begin{fulllineitems}
\phantomsection\label{classes:bopcal.BOPfox.get_name}\pysiglinewithargsret{\bfcode{get\_name}}{}{}
Returns the name of the calculator (string).

\end{fulllineitems}

\index{inspect\_modelsbx() (bopcal.BOPfox method)}

\begin{fulllineitems}
\phantomsection\label{classes:bopcal.BOPfox.inspect_modelsbx}\pysiglinewithargsret{\bfcode{inspect\_modelsbx}}{\emph{key}, \emph{vtype=\textless{}type `str'\textgreater{}}}{}
Checks the value of key in modelsbx.

\end{fulllineitems}

\index{write\_infox() (bopcal.BOPfox method)}

\begin{fulllineitems}
\phantomsection\label{classes:bopcal.BOPfox.write_infox}\pysiglinewithargsret{\bfcode{write\_infox}}{\emph{**kwargs}}{}
Writes infox.bx from input parameters, atomsbx and bondsbx
compatible.

\end{fulllineitems}

\index{write\_infox\_new() (bopcal.BOPfox method)}

\begin{fulllineitems}
\phantomsection\label{classes:bopcal.BOPfox.write_infox_new}\pysiglinewithargsret{\bfcode{write\_infox\_new}}{\emph{**kwargs}}{}
Writes infox.bx from input parameters, models.bx compatible.

\end{fulllineitems}


\end{fulllineitems}

\phantomsection\label{classes:module-bopmodel}\index{bopmodel (module)}\index{atomsbx (class in bopmodel)}

\begin{fulllineitems}
\phantomsection\label{classes:bopmodel.atomsbx}\pysiglinewithargsret{\strong{class }\code{bopmodel.}\bfcode{atomsbx}}{\emph{**kwargs}}{}
Defines an atoms.bx object for BOPfox.
\index{get\_atom() (bopmodel.atomsbx method)}

\begin{fulllineitems}
\phantomsection\label{classes:bopmodel.atomsbx.get_atom}\pysiglinewithargsret{\bfcode{get\_atom}}{}{}
Returns the element in the atoms.bx object

\end{fulllineitems}

\index{get\_mass() (bopmodel.atomsbx method)}

\begin{fulllineitems}
\phantomsection\label{classes:bopmodel.atomsbx.get_mass}\pysiglinewithargsret{\bfcode{get\_mass}}{}{}
Returns the mass of the element

\end{fulllineitems}

\index{get\_version() (bopmodel.atomsbx method)}

\begin{fulllineitems}
\phantomsection\label{classes:bopmodel.atomsbx.get_version}\pysiglinewithargsret{\bfcode{get\_version}}{}{}
Returns the version of the atoms.bx object

\end{fulllineitems}

\index{rattle() (bopmodel.atomsbx method)}

\begin{fulllineitems}
\phantomsection\label{classes:bopmodel.atomsbx.rattle}\pysiglinewithargsret{\bfcode{rattle}}{\emph{var='all'}, \emph{factor='random'}, \emph{maxf=0.1}}{}
Generate a modified atomsbx with parameters defined in var modified 
by factor.
\begin{quote}\begin{description}
\item[{Parameters}] \leavevmode
\end{description}\end{quote}
\begin{itemize}
\item {} 
\emph{var}: dictionary of key, constraints

\item {} 
\emph{factor}: float

\end{itemize}

\end{fulllineitems}

\index{write() (bopmodel.atomsbx method)}

\begin{fulllineitems}
\phantomsection\label{classes:bopmodel.atomsbx.write}\pysiglinewithargsret{\bfcode{write}}{\emph{filename='atoms.bx'}, \emph{update=False}}{}
Writes out atomsbx object to file.
The keyword update does not necessarily mean that atomsbx
is updated. It just means that succeeding entries will not
have the header `Version'.

\end{fulllineitems}


\end{fulllineitems}

\index{bondsbx (class in bopmodel)}

\begin{fulllineitems}
\phantomsection\label{classes:bopmodel.bondsbx}\pysiglinewithargsret{\strong{class }\code{bopmodel.}\bfcode{bondsbx}}{\emph{**kwargs}}{}
Defines a bonds.bx object for BOPfox.
\index{get\_bond() (bopmodel.bondsbx method)}

\begin{fulllineitems}
\phantomsection\label{classes:bopmodel.bondsbx.get_bond}\pysiglinewithargsret{\bfcode{get\_bond}}{}{}
Returns the elements in the bonds object

\end{fulllineitems}

\index{get\_version() (bopmodel.bondsbx method)}

\begin{fulllineitems}
\phantomsection\label{classes:bopmodel.bondsbx.get_version}\pysiglinewithargsret{\bfcode{get\_version}}{}{}
Returns the version of the bonds object

\end{fulllineitems}

\index{rattle() (bopmodel.bondsbx method)}

\begin{fulllineitems}
\phantomsection\label{classes:bopmodel.bondsbx.rattle}\pysiglinewithargsret{\bfcode{rattle}}{\emph{var='all'}, \emph{factor='random'}, \emph{maxf=0.1}}{}
Generate a modified bondsbx with parameters defined in var modified 
by factor.
\begin{quote}\begin{description}
\item[{Parameters}] \leavevmode
\end{description}\end{quote}
\begin{itemize}
\item {} 
\emph{var}: dictionary of key, constraints

\item {} 
\emph{factor}: float

\end{itemize}

\end{fulllineitems}

\index{write() (bopmodel.bondsbx method)}

\begin{fulllineitems}
\phantomsection\label{classes:bopmodel.bondsbx.write}\pysiglinewithargsret{\bfcode{write}}{\emph{filename='bonds.bx'}, \emph{update=False}}{}
Writes bondsbx object to file
update means that you read the current bonds.bx file
and write it with bopbonds

\end{fulllineitems}


\end{fulllineitems}

\index{modelsbx (class in bopmodel)}

\begin{fulllineitems}
\phantomsection\label{classes:bopmodel.modelsbx}\pysiglinewithargsret{\strong{class }\code{bopmodel.}\bfcode{modelsbx}}{\emph{**kwargs}}{}
Defines a modelsbx object in BOPfox
\index{bond\_parameters\_to\_functions() (bopmodel.modelsbx method)}

\begin{fulllineitems}
\phantomsection\label{classes:bopmodel.modelsbx.bond_parameters_to_functions}\pysiglinewithargsret{\bfcode{bond\_parameters\_to\_functions}}{\emph{bondlist='all'}, \emph{variables=None}}{}
Convert list of parameters in bondsbx to corresponding function 
(see :mod functions)

\end{fulllineitems}

\index{rattle() (bopmodel.modelsbx method)}

\begin{fulllineitems}
\phantomsection\label{classes:bopmodel.modelsbx.rattle}\pysiglinewithargsret{\bfcode{rattle}}{\emph{var='all'}, \emph{factor='random'}, \emph{maxf=0.1}}{}
Generate a modified bondsbx with parameters defined in var modified 
by factor.
\begin{quote}\begin{description}
\item[{Parameters}] \leavevmode
\end{description}\end{quote}
\begin{itemize}
\item {} 
\emph{var}: list of dictionary of key, constraints
\begin{quote}
\begin{description}
\item[{e.g. {[}\{`bond':{[}'Fe','Fe'{]},'sssigma':{[}True,True,True{]}\}}] \leavevmode
,\{`atom':{[}'Fe'{]}, `onsitelevels':{[}True{]}\}{]}

\end{description}
\end{quote}

\item {} 
\emph{factor}: float

\end{itemize}

\end{fulllineitems}


\end{fulllineitems}



\section{Miscellaneous}
\label{classes:module-beta}\label{classes:miscellaneous}\index{beta (module)}\index{Beta (class in beta)}

\begin{fulllineitems}
\pysiglinewithargsret{\strong{class }\code{beta.}\bfcode{Beta}}{\emph{**kwargs}}{}
Defines a set of bond integrals for constructing the hamiltonian.

It reads the data points from a file stored in pathtobetas. 
A set of functions in fitfuncs is parametrized with respect to the data.

The betas file should be named as 
{[}struc{]}\_{[}ele1{]}{[}ele2{]}\_{[}basis{]}\_{[}betatype{]}.betas

See the folder betas in the examples folder for reference.
\begin{quote}\begin{description}
\item[{Parameters}] \leavevmode\begin{itemize}
\item {} 
\emph{structure}: str
\begin{quote}

structure from which the bond integrals were derived
\end{quote}

\item {} 
\emph{bondpair}: tuple
\begin{quote}

pair of chemical symbols
\end{quote}

\item {} 
\emph{basis}: str
\begin{quote}

basis used for downfolding
\end{quote}

\item {} 
\emph{betatype}: str
\begin{quote}

can be unscreened, overlap, loewdin or other beta type
\end{quote}

\item {} 
\emph{valence}: tuple
\begin{quote}

pair of valency, e.g. \code{sd}
\end{quote}

\item {} 
\emph{pathtobetas}: str
\begin{quote}

path to betas file
\end{quote}

\end{itemize}

\end{description}\end{quote}
\index{read\_betas() (beta.Beta method)}

\begin{fulllineitems}
\pysiglinewithargsret{\bfcode{read\_betas}}{}{}
Reads in bond integrals from a betamaker output file(.betas) 
saved in pathtobeta. If not provided, will read from folder:
/betas. The filenames must follow the format: 
{[}structure{]}\_{[}elementA{]}{[}elementB{]}\_{[}basis{]}\_{[}betatype{]}.betas

\end{fulllineitems}

\index{set\_fitfuncs() (beta.Beta method)}

\begin{fulllineitems}
\pysiglinewithargsret{\bfcode{set\_fitfuncs}}{\emph{fitfuncs}}{}
Set fitting function , default(sum\_exponential)

\end{fulllineitems}

\index{zip\_betas() (beta.Beta method)}

\begin{fulllineitems}
\pysiglinewithargsret{\bfcode{zip\_betas}}{}{}
Returs a dictionary of the bond integrals

\end{fulllineitems}


\end{fulllineitems}

\renewcommand{\indexname}{Index}
\printindex
\end{document}
